\documentclass[usenatbib,usegraphicx,letterpaper]{mn2e}
%\documentclass[usenatbib,letterpaper]{mn2e}
\usepackage[totalwidth=480pt,totalheight=680pt]{geometry}

\usepackage{amssymb}
\usepackage{epsfig}
\usepackage{amsmath}

%--------- journals
\newcommand{\mnras}{MNRAS~}
\newcommand{\apj}{Ap. J.~}

%--------- general
\newcommand{\beq}{\begin{equation}}
\newcommand{\eeq}{\end{equation}}
\newcommand{\ben}{\begin{enumerate}}
\newcommand{\een}{\end{enumerate}}
\newcommand{\bit}{\begin{itemize}}
\newcommand{\eit}{\end{itemize}}
\newcommand{\beqray}{\begin{eqnarray}}
\newcommand{\eeqray}{\end{eqnarray}}

%--------- gaps
\newcommand{\vone}{V_{1}}
\newcommand{\vtwo}{V_{2}}
\newcommand{\vonetwo}{V_{2/1}}
\newcommand{\monetwo}{m_{12}}
\newcommand{\bcg}{\mathrm{BCG}}
\newcommand{\bsg}{\mathrm{BSG}}
\newcommand{\ndim}{N_{\mathrm{dim}}}
\newcommand{\sigmavred}{\sigma_v^{\mathrm{red}}}
\newcommand{\resid}{\delta\,\ln M}
\newcommand{\phirand}{\Phi_{\mathrm{rand}}}
\newcommand{\phiglo}{\Phi_{\mathrm{global}}}
\newcommand{\ffos}{F_{\mathrm{fos}}}

%--------- Halo Catalog
\newcommand{\vpeak}{V_{\mathrm{max}}^{\mathrm{peak}}}
\newcommand{\vmax}{V_{\mathrm{max}}}
\newcommand{\vacc}{V_{\mathrm{max}}^{\mathrm{acc}}}
\newcommand{\vzero}{V_{\mathrm{max}}^{\mathrm{z=0}}}
\newcommand{\vsub}{V_{\mathrm{sub}}}
\newcommand{\vl}{V_{\mathrm{L}}}
\newcommand{\ngal}{n_{g}}
\newcommand{\msun}{M_\odot}
\newcommand{\nh}{n_{h}}
%\newcommand{\mhost}{M_{\mathrm{host}}}
\newcommand{\mhost}{M}
\newcommand{\lnmhost}{\ln M}
\newcommand{\mfit}{M_{\mathrm{fit}}}
\newcommand{\lnmfit}{\mathrm{ln}M_{\mathrm{fit}}}
\newcommand{\mri}{M_{\mathrm{r}}^{i}}
\newcommand{\mr}{M_{\mathrm{r}}}
\newcommand{\sigmai}{\sigma^{i}}
\newcommand{\veff}{V_{\mathrm{eff}}}
\newcommand{\vbolshoi}{V_{\mathrm{Bolshoi}}}
\newcommand{\lthresh}{M^{\mathrm{thr}}_{\mathrm{r}}}
\newcommand{\linit}{M^{\mathrm{init}}_{\mathrm{r}}}
\newcommand{\lscatter}{M^{\mathrm{scatter}}_{\mathrm{r}}}
\newcommand{\lmatch}{M^{\mathrm{match}}_{\mathrm{r}}}
\newcommand{\nrank}{n_{\mathrm{rank}}}

%--------- Cosmology
\newcommand{\Omegam}{\Omega_{M}}
\newcommand{\Omegab}{\Omega_{b}}
\newcommand{\littleh}{h}
\newcommand{\littlehsq}{h^{2}}
\newcommand{\tilt}{n_{s}}
\newcommand{\sigmaeight}{\sigma_{8}}
\newcommand{\lcdm}{\Lambda\mathrm{CDM}}

%--------- Units
\newcommand{\hmpc}{h^{-1}\mathrm{Mpc}}
\newcommand{\hmpcinv}{h/\mathrm{Mpc}}
\newcommand{\hkpc}{h^{-1}\mathrm{kpc}}
\newcommand{\hmsun}{h^{-1}\mathrm{M}_{\odot}}

%--------- Miscellaneous
\newcommand{\ith}{i^{\mathrm{th}}}
\newcommand{\dd}{\mathrm{d}}


\bibliographystyle{mn2e}


\begin{document}

\title[Mind the Gap]
{Mind the Gap: New Tests of Galaxy-Halo Abundance Matching with Galaxy Groups}

\author[A.P. Hearin et al.]
{Andrew P. Hearin$^{1,2}$,
Andrew R. Zentner$^2$,
Andreas A. Berlind$^3$, 
Jeffrey A. Newman$^2$ \\
$^1$ Fermilab Center for Particle Astrophysics, 
Fermi National Accelerator Laboratory, Batavia, Illinois 60510-0500; aph15@pitt.edu \\
$^2$ Department of Physics and Astronomy \& Pittsburgh Particle physics, Astrophysics and Cosmology Center (PITT PACC),\\ 
University of Pittsburgh, Pittsburgh, PA 15260; zentner@pitt.edu, janewman@pitt.edu\\
$^3$ Department of Physics and Astronomy, Vanderbilt University, Nashville, TN;
a.berlind@vanderbilt.edu
}

\maketitle

%
\begin{abstract} 
%
{\bf MODIFY the abstract.  Short statements.  Make explicit 
statements.  In particular, we CAN make explicit 
statements about SHAM assignments (All SHAM models 
have some tension with the data? vpeak and vzero assignments 
have tension even with the group abundance!).  We CAN make 
explicit statements about correlations between satellite luminosities 
or correlations between satellite and central luminosities.  
We CAN make explicit statements about the CLF assumptions. 
These may be tentative (and you can use the word tentative), 
but they can be clear, concise statements.  Use the abstract to 
attract readers!  Try to take a stab at this before I go for it.  
This is an important skill!}
We employ a mock catalog of galaxy groups constructed via subhalo
abundance matching (SHAM) to provide several new tests of the SHAM
prescription for the galaxy-dark matter connection. By comparing our
mock catalogs to galaxy groups observed in the Sloan Digital Sky
Survey (SDSS), we find that abundance matching faithfully reproduces
the abundance of galaxy groups as a function of richness ($g(N)$) as
well as the relation between group richness, $N,$ and line-of-sight
velocity dispersion, $\sigma_{v}.$ Additionally, we find that
observations of $g(N)$ may be a promising way to constrain models of
the stellar mass stripping of satellite galaxies. While the global
luminosity function (LF) of galaxies in our mock catalog exactly
matches that of the SDSS catalog by construction, we find an
intriguing discrepancy between the observed and predicted field
(group) galaxy luminosity function, with the SHAM prediction for the
LF of field (group) galaxies being systematically too dim (bright). We
also test the SHAM prediction for the abundance of galaxy groups a
function of magnitude gap, $\monetwo,$ defined as the difference
between the r-band absolute magnitude of the two brightest group
members. The SHAM prediction for gap abundance  is in very good
agreement with the data, constituting a new success of the abundance
matching prescription. Moreover, our results suggest that the
$\monetwo$ abundance is a statistic that is well-suited to constrain
the intrinsic scatter in the map between dark matter halo mass and
galaxy luminosity, thereby providing complementary constraints on SHAM
to $g(N)$.  Finally, we generalize existing data randomization
techniques to provide a new way to discriminate between competing
hypotheses for how galaxies are arranged into groups. We find the
hypothesis that the luminosity gap is constructed via random draws
from a universal  luminosity function to provide a poor description of
the data, contradicting recent claims in the literature. We further
utilize these data randomization techniques to demonstrate that the
observed gap abundance in rich groups is consistent with the
hypothesis that the satellite galaxy LF is determined solely by the
luminosity of the central galaxy. However, this hypothesis badly fails
to describe the gap abundance in groups with only a few members,
suggesting the possibility  that the satellite LF may require
conditioning from a variable in addition to $L_{cen}$ in poor groups,
or that draws from $\Phi_{sat}(L)$ and $\Phi_{cen}(L)$ are correlated.
%
\end{abstract}
%

\begin{keywords}
{cosmology: theory -- galaxies: structure -- galaxies: evolution}
\end{keywords}

\maketitle

%---------------------------
\section{Introduction}
\label{section:introduction} 
%--------------------------- 

The centers of dark matter halos are the natural sites for galaxy
formation, as these are the locations of the deepest gravitational
potential wells in the universe \cite[e.g.,][]{white_rees78}. 
The development of a theory of galaxy
formation that encompasses the complex array of physical processes
known to contribute to cosmic structure formation is one of the
fundamental goals of astrophysics, and ennumerating the connection between 
galaxies and dark matter halos may help to establish the foundations of any 
such theory. Additionally, such a connection can serve as 
an empirical link between large-scale survey data and theoretical 
predictions.  Furthermore, our contemporary theory of cosmology, $\lcdm,$
makes precise, quantitative predictions for the distribution of dark
matter in the universe over a wide range of scales, and so establishing 
the galaxy-dark matter connection is a key step toward unlocking the
predictive power of $\lcdm.$

One of the most commonly used techniques for connecting dark matter
halos to galaxies is subhalo abundance matching (SHAM). The
fundamental tenet of all SHAM models is that there is a monotonic
mapping between some elementary property of galaxies (usually
luminosity or stellar mass) and an elementary property of halos. SHAM
models determine this mapping through the implicit relation defined by
matching the predicted abundance of halos with the observed abundance
of galaxies. When used in concert with numerical simulations of
cosmological structure formation, abundance matching techniques have
been shown to predict accurately galaxy clustering statistics
\citep{kravtsov_etal04,tasitsiomi_etal04,conroy_etal06,behroozi_etal10,moster_etal10},
the Tully-Fisher relation \citep{trujillo-gomez_etal11}, and the
conditional stellar mass function \citep{reddick_etal12}.  
To date, the vast majority of tests of the SHAM algorithm have 
been based upon clustering.  We will present new tests of the 
SHAM method.

In our contemporary model of cosmology, $\lcdm,$ gravitationally
self-bound structures form hierarchically, with tiny peaks in the
initial cosmic density field collapsing into very small dark matter
halos that gradually merge together to form groups and clusters of
galaxies. Galaxy groups are thus interesting environments for testing
theories of structure formation in general, and the galaxy-halo
connection in particular. Indeed, the influence of the group
environment on galaxy properties has a long history and has 
received considerable attention in the recent literature,
\cite[e.g.,][]{yang_etal05,zandivarez_etal06,robotham_etal06,yang_etal08,yang_etal09,tinker_etal11a,gerke_etal12}. 

In this paper we investigate SHAM predictions for the assembly of
galaxies into groups and test these predictions against galaxy groups
observed in SDSS. We briefly describe the SDSS catalog of galaxy
groups against which we compare our predictions in
\S~\ref{section:data}. We conduct our study of abundance matching by
first constructing a SHAM-based mock catalog of galaxies and then
applying a group-finding algorithm on the mock catalog that is identical 
to that used to find groups in the SDSS galaxy sample. Details of the SHAM
algorithm, such as the particular halo property used in the abundance
matching, vary in the literature, and we have investigated several
choices for the specific implementation of the SHAM algorithm with the
aim of determining which methods best reproduce the observed
properties of galaxy groups. We describe our methods for constructing
our mock catalogs in \S~\ref{section:mocks}, and provide a detailed
description of some novel features of our SHAM implementation in
Appendix A. 

By comparing properties of the mock and observed galaxy groups we
provide a series of new tests of the abundance matching prescription
connecting galaxies to dark matter halos. One of the fundamental
properties of any catalog of groups is the {\em multiplicity
function}, the abundance of groups as a function of {\em richness},
$N,$ the number of members in a group. Previous studies of galaxy
group catalogs \citep{berlind_etal06,vale_ostriker06} have
demonstrated that measurements of the group multiplicity function
$g(N)$ may contain valuable information about how galaxies populate
dark matter halos. Motivated by this, in \S~\ref{subsection:gn} we
compare the observed multiplicity function to that which is exhibited
by our mocks. The line-of-sight velocity dispersion of group members,
$\sigma_v,$ encodes information about the mass of the group
\cite[e.g.,][]{becker_etal07}; in \S~\ref{subsection:gn} we also
investigate the ability of SHAM to predict the correct relationship
between group richness and $\sigma_v.$ While the abundance matching
algorithm permits an exact reproduction of the observed galaxy
luminosity function by construction, SHAM need not necessarily
reproduce the luminosity function of a subsample of galaxies that has
been conditioned on some property. In \S~\ref{subsection:groupfield}
we test the SHAM prediction for the galaxy luminosity function
conditioned on whether or not the galaxies are members of a group.

One galaxy group property that has received attention from a rapidly
growing body of literature is the luminosity gap, $\monetwo,$ the
difference in r-band absolute magnitude between the two most luminous
members of a galaxy group. Significant investigation has been focused on a
class of systems known as {\em fossil groups}, usually defined
as an X-ray bright ($L_{X,bol}>10^{42}$erg/s) group of galaxies with
$\monetwo \geq 2.$ The prevailing theoretical paradigm for 
fossil group formation is that these systems 
have evolved quiescently for a significant fraction
of a Hubble time, during which dynamical friction has had sufficient time to cause
the biggest satellites to merge with the central galaxy, resulting in
a massive, bright central galaxy with few bright satellites
\citep{ponman94,jones_etal03,donghia_etal05,donghia_etal07,dariush_etal07,vonbendabeckmann_etal08,milosavljevic_etal06,vikhlinin_etal99,miller_etal11,labarbera_etal12,tavasoli_etal11}. 
For a recent paper on fossil groups that includes an excellent review of
the history of their study, we refer the interested reader to
\citet{harrison_etal12}. In \S~\ref{subsection:pgap} we study
$\Phi(\monetwo),$ the abundance of our mock and observed groups as a
function of $\monetwo,$ finding that this statistic has the potential 
to constrain the manner in which galaxies populate dark
matter halos. This conclusion is consistent with previous work
\citep{skibba_etal07}, showing that the relative brightness of the
central galaxy in a group and its brightest satellite is influenced by
parameters governing the conditional luminosity function.

In a paper studying the same galaxy group catalog we use here,
\citet{paranjape_sheth11} found that the observed $\Phi(\monetwo)$ is
consistent with the hypothesis that the brightness distribution of
galaxy group members is determined by a set of random draws from a
universal luminosity function. This finding implies that, for a galaxy
population of a given luminosity function, the gap abundance is
uniquely determined by knowledge of the abundance of groups as a
function of richness. This result is particularly surprising in light
of recent results \citep{hearin_etal12} demonstrating that the
magnitude gap contains information about group mass that is
independent of richness. A possible resolution to this apparent
discrepancy was recently pointed out in \citet{more12}: the global gap
abundance $\Phi(\monetwo)$ is a mass function-weighted sum over the
mass-conditioned gap abundance, $\Phi(\monetwo|M),$ and so it is
possible in principle that the magnitude gap depends on both mass and richness in
such a way that the mass function-weighting washes out any
statistically-significant mass dependence in the global
$\Phi(\monetwo).$ 

In \S~\ref{section:mcs} we show that the global gap abundance
exhibited by galaxy groups in SDSS is inconsistent with the random
draw hypothesis, contradicting the conclusions in
\citet{paranjape_sheth11}. As discussed in Appendix B, we find that a
subtle systematic error in the \citet{paranjape_sheth11} measurement
of $\Phi(\monetwo)$ is responsible for the difference in our
conclusions. Additionally, we generalize these data randomization
techniques and test an alternative hypothesis of galaxy group
formation, namely that the satellite luminosity function need only be
conditioned on the brightness of the central galaxy in order to
account for the observed gap abundance.  We discuss our results and
compare them to those in the existing literature in
\S~\ref{section:discussion}, and conclude with a brief overview of our
primary findings in \S~\ref{section:conclusion}.


%---------------------------
\section{Data}
\label{section:data}
%---------------------------

We study galaxy group properties in a volume-limited catalog of 
groups identified in Sloan Digital Sky Survey (SDSS) 
Data Release 7 \cite[][DR7 hereafter]{sdss_dr7}
using the algorithm described in \cite{berlind_etal06}. 
This catalog is an update of the \citet{berlind_etal06} group 
catalog (which was based on SDSS Data Release 3). 
The galaxies in this sample are all members of the Main
Galaxy Sample of SDSS DR7. Groups in this galaxy catalog are
identified via a redshift-space friends-of-friends algorithm that
takes no account of member galaxy properties beyond their redshifts, 
positions on the sky, and relative velocities. Our groups are constructed from galaxies in
a volume-limited spectroscopic sample ($\veff \simeq 5.8 \times10^{6}(\hmpc)^3$) in the redshift range 
$0.02 \le z \le 0.068$ with r-band absolute magnitude $M_r - 5\log h < -19$. 
We refer to this catalog as the ``Mr19'' group catalog. Each of the
$6439$ groups in the Mr19 catalog contains $N \geq 3$ members. We refer
the reader to \citet{berlind_etal06} for further details on the  group
finding algorithm. 

Fiber collisions occur when the angular separation between two or more
galaxies is closer than the minimum separation permitted by the finite
width of the optical fibers used to measure galaxy spectra 
\citep[see][and references therein]{guo_etal12}. We briefly note here that
fiber collisions in DR7 are treated differently than the catalog based
on DR3 data. As we will see in \S~\ref{section:mcs}, this different
treatment has important consequences for the measurement of magnitude
gaps. In Appendix B we discuss these differences in detail and argue
that the DR3 treatment induces systematic errors in magnitude gap
measurements that can be avoided if fiber collisions are instead
modeled as they are in DR7.

%---------------------------
\section{Mock Catalogs}
\label{section:mocks}
%---------------------------


We compare the SDSS DR7 group data to a mock catalog of 
galaxy groups based on the Bolshoi N-body simulation \citep{klypin_etal11}. 
The Bolshoi simulation models the cosmological growth of structure 
in a cubic volume $250\,\littleh^{-1}\mathrm{Mpc}$ on a side within a 
standard $\lcdm$ cosmology with total matter density $\Omegam=0.27$, 
Hubble constant $\littleh=0.7$, 
power spectrum tilt $\tilt=0.95$, 
and power spectrum normalization $\sigmaeight=0.82$. 
The Bolshoi data are available at {\tt http://www.multidark.org} and we refer the 
reader to \citet{riebe_etal11} for additional information.
Our analysis requires reliable identification of self-bound 
subhalos within the virial radii of distinct halos. We utilize the 
{\tt ROCKSTAR} \citep{behroozi_etal11} halo finder in order to identify 
halos and subhalos within Bolshoi. 

We utilize the subhalo abundance matching (SHAM) technique to associate 
galaxies with dark matter halos. Although abundance matching is widely 
used to construct mock galaxy catalogs 
\cite[e.g.,][]{kravtsov_etal04,tasitsiomi_etal04,conroy_etal06,watson_etal11,hearin_etal12,reddick_etal12}, 
our particular implementation of SHAM is novel and so we describe it 
in detail in Appendix A. In this section we provide a brief sketch of our 
SHAM prescription and review the primary advantages of our implementation.


SHAM models assume a monotonic relationship between the stellar masses 
of galaxies and the maximum circular speeds of test particles 
within their host dark matter halos, $\vmax\equiv\mathrm{max}\left[\sqrt{GM(<r)/r}\right]$, 
where $r$ is the distance from the halo center and $M(<r)$ is the halo mass 
contained within a distance $r$ of the halo center. 
In practice, stellar masses must be inferred from imaging and 
spectroscopic data.  Inferring galaxy stellar masses is non-trivial, 
so in practice galaxy luminosities are often used to associate galaxies 
with halos using SHAM, though this may introduce important biases \cite[e.g.,][]{simha_etal12}.  
It is most common to construct SHAM assignments for SDSS data using 
the r-band absolute luminosities of the galaxies as rough proxies for stellar mass.  
We follow this approach in this paper.  


Assuming a monotonic relationship 
between $\vmax$ and r-band absolute magnitude $M_r$, the SHAM 
galaxy-halo assignment follows by requiring the number density of galaxies 
of a given $M_r$ to be equal to the number density of halos of the $\vmax$ 
to which they are assigned.  As a further complication, subhalos within 
host dark matter halos evolve significantly due to interactions within 
the dense environments of the larger host halos.  As a result, the present 
values of $\vmax$, which we denote $\vzero$, may be a poor proxy for stellar 
masses or r-band luminosities.  It is now common practice to assign luminosities 
to subhalos based on their values of $\vmax$ {\em evaluated at the time 
at which they merged into their distinct host halos}, $\vacc$.  Often, 
a halo can be significantly affected by interactions prior to entering 
the virialized region of a distinct host halo, so it is also interesting 
to explore using the maximum value of $\vmax$ ever attained by a subhalo 
as the stellar mass/luminosity proxy, $\vpeak$.  

The SHAM assignment of r-band luminosities to halos and subhalos occurs through 
the implicit relation 
%
\beq
\label{eq:lv}
\ngal(<M_r)=\nh(>\vl), 
\eeq
%
where $\ngal(<\mr)$ is the number density of observed galaxies with r-band 
magnitudes brighter than $\mr$ \citep{blanton_etal05}, 
and $\nh(>\vl)$ is the predicted number density of dark matter halos 
and subhalos with assigned circular speeds $>\vl$. 
As we mentioned in the previous paragraph, the circular speeds assigned 
to subhalos are often not their circular speeds at the time of 
observation.  The quantity $\vl$ is evaluated as 
%
\begin{eqnarray}
\label{eq:vassign}
\quad \quad \vl & = & \vzero \quad \mathrm{(host\ halos)} \nonumber \\
    & = & \vsub   \quad \ \mathrm{(subhalos)} \nonumber
\end{eqnarray}
%
where $\vsub=\vzero$ if one chooses to use $\vzero$ to describe 
the luminosities of subhalos, $\vsub=\vacc$ if one chooses to 
use the maximum circular velocity at accretion for subhalos, 
and $\vsub=\vpeak$ if on chooses to use the maximum value of 
$\vmax$ ever attained by the subhalo.  There is significant 
empirical support for SHAM models premised on 
$\vacc$ \cite[e.g.,][]{conroy_etal06} as well as $\vpeak$ \cite[e.g.,][]{reddick_etal12}; 
however, we explore each of these three choices. Throughout this paper, 
we refer to mock catalogs constructed using $\vzero$ 
as ``SHAM0'' catalogs, those built with $\vacc$ as ``SHAMacc'', 
and catalogs constructed using $\vpeak$ as ``SHAMpeak''.


The SHAM construction summarized in Eq.~(\ref{eq:lv}) enables luminosities 
to be assigned to dark matter halos 
in a manner that can match {\em any} observed galaxy luminosity function. 
Tests of SHAM consist of comparing observational data with independent 
predictions.  The advantages of SHAM-like models is that they are 
simple, they embody the fundamental theoretical prejudice that dark 
matter halos that represent deeper gravitational wells should host 
larger (more luminous) galaxies, and such models describe galaxy 
clustering over a range of redshifts remarkably well 
\cite[e.g.,][]{kravtsov_etal04,tasitsiomi_etal04,conroy_etal06,behroozi_etal12}.  


In practice, some scatter between $\vmax$ and $M_r$ is often introduced into the 
basic SHAM assignments. The scatter accounts for the fact that galaxy formation 
is a complex process, so a single halo parameter cannot specify a stellar 
mass (or luminosity).  Perhaps more importantly, scatter brings SHAM predictions 
into better agreement with galaxy clustering statistics 
\citep[e.g.,][]{tasitsiomi_etal04,behroozi_etal10,reddick_etal12}. We investigate 
the influence of scatter on our results using three different models of 
scatter between circular velocity and absolute magnitude. 
Our fiducial model, whose construction is described in detail in Appendix A, 
is designed to be similar to the model explored in \citet{trujillo-gomez_etal11} 
and has $0.2$dex of scatter at the faint end 
({\bf ARZ: perhaps a parenthetical definition of 
``faint end'' would be useful here.}) 
and $0.15$dex of scatter at the bright end ({\bf ARZ: same for bright end}). 
{\bf [ARZ: I thought the following statement might help here, please evaluate 
it as you see fit.]} 
Our model cannot be precisely the same as that of \citet{trujillo-gomez_etal11} 
because we use a different SHAM algorithm to incorporate scatter. 
Second, we explore a model which has a constant scatter of $0.1$dex.  
We will refer to as our ``alternate'' scatter model. 
Third, we consider models with no scatter between $M_r$ and $\vmax.$


A detail of SHAM implementation is that a specific choice must be made 
for the galaxy luminosity function 
to use for the SHAM assignments.  A common and convenient 
choice is to use a fit to observed luminosity functions, 
such as that provided by \citet{blanton_etal05}.  
However, in order to ensure that our results are not affected by 
the residuals of any such fit, the mock catalogs that we 
construct match the global luminosity function of the 
Mr19 galaxies {\em exactly}.  Enforcing the requirement that the SHAM 
galaxy catalog have a luminosity function that matches the 
observed luminosity function exactly complicates the 
introduction of scatter into the SHAM prescription. 
To enforce the observed galaxy luminosity function 
on our SHAM assignments with scatter in the 
$\vl$-$M_r$ relation, we use a novel implementation 
of SHAM, the details of which are given in Appendix A.  
The important features of this implementation are as 
follows.
%
\ben
\item[1.] Our mock galaxy catalog has a luminosity function 
that matches the observed Mr19 luminosity function {\em exactly}, 
by construction.
\item[2.] The amount of scatter in the $\vl$-$M_r$ relation can be specified simply, so that 
implementing SHAM assignments with differing amounts of scatter is straightforward.
\item[3.] Even when the above two requirements are met, the algorithm is very fast, lending itself 
to applications that require the construction of a large number of mock catalogs.  This advantage 
is significant compared to, for example, the algorithm of \citet{trujillo-gomez_etal11}.
\een
%

Once galaxies with r-band luminosities have been assigned to dark matter halos and subhalos, 
we proceed to find groups. The galaxies inherit the positions and velocities of their host 
halos and we have identified groups using the same algorithm that was applied to the observational data. 
This guarantees that our mock groups are subject to the same redshift-space projection effects as 
the Mr19 catalog.

Once groups have been found, we introduce fiber collisions to our mock galaxies as follows. 
For each mock group of richness $N$ we randomly select a Mr19 group with a similar richness. 
If the number of fiber-collided members of the randomly selected group is $N_{\mathrm{fc}},$ 
then we randomly choose $N_{\mathrm{fc}}$ of the members of the mock group and assign fiber 
collisions to these members. This procedure ensures that the fraction of fiber-collided galaxies 
in our mock groups {\em scales with richness in the same way as it does in the data}. Fiber collisions 
play an important role in our definition of the magnitude gap, as discussed in \S~\ref{subsection:pgap}.
{\bf [ARZ: Perhaps we should mention this caveat simply so that we do not appear naive.]} 
Of course, this correction does not account for the spatial biases of fiber collisions, but we 
do not consider spatial clustering in the current study.

%
%---------------------------
\section{New Tests of Abundance Matching}
\label{section:prediction}
%---------------------------

In this section, we scrutinize the abundance matching prescription 
for the mapping between galaxies and dark matter halos in several novel ways.  
In \S~\ref{subsection:gn}, we consider the number density of groups 
as a function group richness, in \S~\ref{subsection:groupfield} we 
study the group and field luminosity functions of galaxies, and in 
\S~\ref{subsection:pgap} we investigate the distribution of galaxy 
luminosities within groups, concentrating significant attention on 
the magnitude gap statistic. As discussed in \S~\ref{section:mocks}, 
we explore three distinct assignments for the effective maximum 
circular velocities to be used in the luminosity assignments of 
subhalos, namely ``SHAMacc'', ``SHAMpeak'', and ``SHAM0.''  
We also analyze SHAM predictions based upon three distinct models for 
the amount of scatter between $M_r$ and circular velocity.  
Throughout this paper we refer to the SHAMacc model (in which the 
maximum circular velocity at the time of accretion is used for subhalos) 
with $0.2$dex of scatter at the faint end and $0.15$dex of 
scatter at the bright end, following \citet{trujillo-gomez_etal11}, 
as our default model.  We refer explicitly to results from the other 
models where relevant.


%---------------------------
\subsection{Multiplicity Function}
\label{subsection:gn}
%---------------------------


%---------------------------------------------------------------------------------------------------
\begin{figure*}
\centering
\includegraphics[width=12.0cm]{FIGS/multiplicity_function_0815.eps}
\includegraphics[width=12.4cm]{FIGS/multiplicity_function_residuals_0815.eps}
\caption{
Group abundance as a function of the number of group members. 
In the top panel we illustrate a comparison of the group multiplicity function $g(N)$ seen 
in theMr19 SDSS catalog (blue diamonds) and that in our fiducial mock catalog (red
triangles). Our fiducial mock was made with abundance matching on $\vacc.$ We also display 
results when the abundance matching is done on $\vpeak$ (top dashed curve) as well as $\vzero$ 
(bottom dashed curve). Error bars for these alternate SHAM models have been suppressed as they 
are very similar to those in our fiducial model. 
In the bottom panel we plot the fractional difference between the predicted and observed g(N) 
for each of these three models. The multiplicity function in our fiducial mock is well within 
$2\sigma$ of the observed $g(N)$ for any richness cut $N\geq2.$ SHAM models using $\vpeak$ 
and $\vzero$ result in large discrepancies that straddle the $\vacc$ result, suggesting 
that $g(N)$ may be a useful statistic to constrain models of stellar mass stripping. All 
models depicted in this figure pertain to our fiducial scatter model, though we find no 
qualitative change to these results when exploring our alternate scatter model, or SHAM models without scatter.
{\bf [ARZ: I'm mentally distraught over the fact that the two panels aren't the same size.  
Changing subject, is it possible to make the symbols any bigger or is this just as good 
as it can get (which is OK with me).  Also, in the upper panel can you plot a line with 
a slope $g(N) \propto N^{-2.5}$ 
in the upper right white space just for reference for the reader?]}
}
\label{fig:gn}
\end{figure*}
%-----------------------------------------------------------------------------------------------------


In this section, we compare the predictions of abundance matching 
for the average number density of groups as a function of the 
number of group members, $N$.  We refer to this abundance 
as the group {\em multiplicity function} 
and represent it as $g(N)$.  The multiplicity function of 
the observed SDSS Mr19 sample is plotted in blue diamonds in the 
top panel of Fig.~\ref{fig:gn}.  The points represent the mean number 
densities of groups, while the errors were computed by bootstrap 
resampling of the group catalogs. For all of the results in this paper, 
our bootstrap errors are computed as the variance over $10^4$ bootstrap 
realizations, where each realization is constructed by randomly 
selecting (with replacement) the same number of objects in the sample.  
In all but the smallest richness bin, the observed multiplicity function 
is consistent with a power-law of $g(N) \propto N^{-2.5}$.

In addition to depicting the SDSS Mr19 group multiplicity function, 
Fig.~\ref{fig:gn} shows how these data may be used to scrutinize 
empirical methods to assign galaxies to halos, such as SHAM. 
To illustrate this, we plot with red triangles the multiplicity 
function prediction of our fiducial SHAMacc mock catalog. The upper (lower) 
dashed curve in the top panel of Fig.~\ref{fig:gn} shows the SHAMpeak 
(SHAM0) prediction for the multiplicity function; error bars for these 
two models have been omitted as they are similar to those from our 
fiducial SHAMacc model. To further facilitate the illustration of the 
potential of $g(N)$ measurements to discriminate between different 
abundance matching prescriptions, in the bottom panel of 
Fig.~\ref{fig:gn} we plot the fractional difference between 
the predicted and observed multiplicity functions. All the points 
appearing in Fig.~\ref{fig:gn} trace models with our 
fiducial amount of scatter between $M_r$ and $\vmax.$


Our fiducial SHAMacc model, is consistent with the observed multiplicity 
function well within $2\sigma$ for any richness cut $N \geq 2$.  On the other hand, 
the SHAM0 assignment with $\vsub=\vzero$ is a significantly poorer description of the 
data.  SHAM0 underestimates the abundances of all groups with $N \geq 5$.  
The significance of the difference between the SDSS group data and the predicted 
SHAM0 multiplicity function is $\simeq 4.9\sigma$ for groups with $N\geq5,$ 
and $\simeq 4.0\sigma$ for $N \geq 10$ groups. 
These discrepancies hold true at similar levels in the alternate 
scatter models we studied. 


This discrepancy may not be surprising 
because SHAM assignments based upon $\vzero$ have 
already been shown to be less effective at describing independent 
data than SHAM assignments based upon $\vacc$ 
\cite[e.g.][]{conroy_etal06,berrier_etal11,watson_etal11,reddick_etal12}.  
More interestingly, our SHAMpeak assignments of galaxies to halos with $\vsub=\vpeak$ 
do not describe the data as well as SHAMacc either.  
SHAMpeak models significantly {\em overestimate} the abundances of rich groups. 
The statistical significance of this discrepancy is $\simeq 3.9\sigma$ 
for groups with $N \geq 5$ members, and $\simeq 2.6\sigma$ for groups with $N \geq 10$.   


These results are interesting as a demonstration that the multiplicity functions 
of groups can be used as valuable statistics with which to constrain the connection 
between dark matter and galaxies. This group multiplicity function test clearly 
indicates that the SHAMacc assignment with $\vsub=\vacc$ is consistent with SDSS 
group data while the alternative SHAM0 and SHAMpeak assignments cannot describe 
the SDSS group data adequately.  In particular, note that the SHAMacc results are 
straddled by the SHAMpeak and SHAM0 results, suggesting that it may be possible 
to use $g(N)$ measurements to constrain models of the mass stripped from satellite 
galaxies \cite[e.g.][]{watson_etal11}. The success of our SHAM models with $\vsub=\vacc$ 
motivates the choice of SHAMacc as our fiducial model.  


We have also investigated the role that scatter plays in the abundance matching 
prediction for $g(N).$ Each of the models depicted in Fig.~\ref{fig:gn} pertain 
to our fiducial model of scatter, which has $0.2$dex of scatter at the faint end 
and $0.15$dex at the bright end. However, we find no qualitative change to our results 
when using our alternate scatter model (which has a constant $0.1$dex of scatter) 
or models with no scatter. In particular, we find that for all the scatter models 
we explored, SHAMpeak models significantly overestimate the abundances of rich groups 
and SHAM0 models significantly underestimate the abundances of rich groups. 
Accordingly, we do not explicitly show the results for the alternative scatter models 
for the sake of brevity.  

In this subsection {\em only}, we study an alternate SDSS catalog that is 
complete to $\mr\leq-20,$ spans the redshift range $0.02<z<0.106,$ and has an 
effective volume of $\veff \simeq 2.3\times10^{7}$ $\mathrm{Mpc}^{3}/h^3$ to 
test the sensitivity of our conclusions to the sample selection. 
We constructed mock catalogs for this Mr20 catalog in exactly the same fashion 
for Mr19, except that we imposed a brightness cut of $\mr\leq-20$ for the mock galaxies. 
Our findings for the Mr20 catalog are the same as for Mr19: 
the observed $g(N)$ is well-fit by a power law with exponent $-2.5$ in all but 
the smallest richness bin; 
the multiplicity function predicted by our fiducial model of the Mr20 groups 
exhibits less than $1\sigma$ discrepancy with the data; 
SHAMpeak (SHAM0) significantly over-(under-)predicts the abundances of rich groups. 



In addition to group abundance as a function of richness, 
we also present results concerning the map between group richness 
and line-of-sight velocity dispersion in Figure \ref{fig:sigmavn}.  
The agreement is quite good except for possibly in the largest $\sigma_v$ bin. 
Plots based on $\vpeak$ or $\vzero$ results look similar, demonstrating that 
this mapping is a generic success of the SHAM paradigm. For reference, we also 
fit the Mr19 data to a power law $N\propto\sigma_{v}^{3},$ and plot the result with 
the black solid line.

%---------------------------------------------------------------------------------------------------
\begin{figure}
\centering
\includegraphics[width=9.0cm]{FIGS/richness_sigmav_0815.eps}
\caption{
Group richness $N$, as a function of line-of-sight velocity 
dispersion $\sigma_v.$ The SDSS Mr19 measurement of this mapping appears 
in blue, the prediction of our fiducial abundance matching model appears 
in red. The black line represents a fit of the Mr19 data to a power 
law $N\propto\sigma_{v}^{3},$ as this is the dependence that would be 
expected if all the galaxy groups are virialized and group mass 
$M\propto N.$ {\bf [ARZ: labels bigger! How exactly did you do this?  What do 
the horizontal error bars signify?]}
}
\label{fig:sigmavn}
\end{figure}
%-----------------------------------------------------------------------------------------------------



%---------------------------
\subsection{Field \& Group Galaxy Luminosity Functions}
\label{subsection:groupfield}
%---------------------------

The SHAM method for assigning galaxies to halos results in mock luminosity 
functions that match observed luminosity functions by construction.  However, 
this procedure does not guarantee agreement between luminosity functions 
that are conditioned on a specific galaxy property or environment.  In this 
section, we consider a simple distinction in galaxy environment based directly 
on our group catalogs.  Specifically, we explore the luminosity functions of galaxies 
conditioned upon whether or not the galaxy is identified as a member of a group. 
We refer to the luminosity function constructed from all galaxies residing in 
groups as $\Phi_{\mathrm{group}}(L)$.  
Likewise, we refer to all galaxies that we do not identify as members of a 
group as ``field'' galaxies and the luminosity function conditioned on the 
galaxy being a member of the field as $\Phi_{\mathrm{field}}(L)$.  
SHAM predictions for $\Phi_{\mathrm{group}}(L)$ and $\Phi_{\mathrm{field}}(L)$ are 
{\em not} guaranteed to match observational determinations of these quantities, so 
this is a test of the allocation galaxies to group and field environments 
by SHAM methods.


Figure~\ref{fig:groupfield} shows comparisons between our predicted, SHAM luminosity 
functions for group galaxies and field galaxies and the corresponding 
observed luminosity functions.  To highlight differences, the quantity 
shown on the vertical axis of all panels in Fig.~\ref{fig:groupfield} is the 
fractional difference $\Delta\Phi/\Phi_{\mathrm{SDSS}}$ between the predictions of the 
SHAM mocks and the observations, 
where $\Delta\Phi(L) \equiv \Phi_{\mathrm{mock}}(L)-\Phi_{\mathrm{SDSS}}(L),$ 
so that points in Fig.~\ref{fig:groupfield} with positive vertical axis values 
correspond to luminosities where SHAM over-predicts the abundance of galaxies of that 
brightness, and conversely for negative values of $\Delta\Phi(L)$.  
Blue diamonds show the error on $\Phi(L)$ for field galaxies, red triangles show the same 
for group galaxies. Group galaxies defined by a richness cut of 
$N_{group} \geq 3$ appear in the left columns, $N_{group}\geq 10$ appear in the 
right columns. Our fiducial model, SHAMacc which abundance matches on $\vacc$, 
appears in the top panels while the alternate, SHAMpeak model which uses $\vsub=\vpeak$, 
appears in the bottom panels; all abundance matching prescriptions in 
Figure~\ref{fig:groupfield} were constructed with our fiducial scatter. 
The error bars on $\Phi_{\mathrm{SDSS}}(L)$ and $\Phi_{\mathrm{mock}}(L)$ have each been 
estimated by bootstrap resampling as in \S~\ref{subsection:gn}. 
We reiterate that the differences shown in this plot are strictly due to 
the separation of field galaxies from group galaxies because our SHAM 
implementation {\em guarantees} that the overall luminosity function matches 
the data {\em exactly}.  

As Fig.~\ref{fig:groupfield} shows, our fiducial SHAMacc model, 
the baseline SHAM procedure which is known to describe galaxy clustering correctly and 
which provided an adequate description of the group multiplicity function in \S~\ref{subsection:gn}, 
systematically {\em overestimates} the abundance of dim, field galaxies and {\em overestimates} 
the abundance of bright, group galaxies.  Notice that this qualitative 
conclusion holds irrespective of the choice for the minimum number of member 
galaxies required in order to have the group members be included in the 
group luminosity function. In fact, the discrepancies grow larger as the number 
of group members grows, emphasizing that the SHAM excesses of bright, group 
galaxies grow more egregious for larger groups (particularly for SHAMpeak).  
These results highlight at least one weakness of 
the SHAM procedure for exploring the connection between galaxies and dark matter halos.  
The widely-used SHAM procedure that adequately describes 
low-redshift galaxy clustering {\em does not allocate luminosities 
to group and field halos in a manner consistent with observations}. 


{\bf [ARZ: Naively, one might expect that the sum of the field and group luminosity 
functions must be the global luminosity function.  This isn't true given your 
definitions. However, this may lead to a minor confusion interpreting Fig.~\ref{fig:groupfield}.  
In particular, one might expect that both the diamonds and triangles cannot be positive (or negative) 
simultaneously.  However, this does occur at some points.  I assume that this occurs because 
the intervening multiplicities (say objects with $N_{group}=2$ in the left panels) make up the 
difference so that the global luminosity function is always fixed? If this is true, you should 
add a sentence or two somewhere to assure the reader of this.]}

%---------------------------------------------------------------------------------------------------
\begin{figure*}
\centering
\includegraphics[width=8.0cm]{FIGS/delta_LF_N3_0815.eps}
\includegraphics[width=8.0cm]{FIGS/delta_LF_N10_0815.eps}
\includegraphics[width=8.0cm]{FIGS/delta_LF_N3_vpeak_0815.eps}
\includegraphics[width=8.0cm]{FIGS/delta_LF_N10_vpeak_0815.eps}
\caption{
Fractional differences between the field and group galaxy
luminosity functions predicted by abundance matching and observed 
in SDSS. The fractional differences for field (group) galaxies appear as 
blue diamonds (red triangles). For this purpose, field galaxies are
defined to be those galaxies that do not reside in groups, and group galaxies 
are defined by the richness cut given in the legend of each panel. 
Results pertaining to our fiducial SHAM model, SHAMacc based on abundance matching with 
$\vsub=\vacc$, appear in the top panels. Results pertaining to SHAM using $\vpeak$, SHAMpeak, 
appear in the bottom panels. The left panels show results for groups with multiplicity 
$N_{group}>2$ while the right panels show results for $N_{group}>9$. 
The group (field) galaxy luminosity functions in all of our 
SHAM mocks is systematically too bright (dim).
 }
\label{fig:groupfield}
\end{figure*}
%-----------------------------------------------------------------------------------------------------

We also explore the influence that scatter between absolute magnitude and halo circular velocity has on 
the SHAM predictions for the group and field luminosity functions. In Figure~\ref{fig:groupfieldscatter}, 
we again plot the fractional difference between the observed and predicted $\Phi_{\mathrm{group}}(L)$ (top left panel) 
and $\Phi_{\mathrm{field}}(L)$ (top right panel), this time comparing results between mocks made with 
different amounts of scatter. The SHAMpeak abundance matching procedure was used for all models 
depicted in Figure~\ref{fig:groupfieldscatter}. Blue diamonds in Fig.~\ref{fig:groupfieldscatter}, 
give results pertaining to our fiducial scatter model, 
which has $0.2$dex of scatter at the faint end and $0.15$dex of scatter at the bright end, 
magenta squares designate our alternate model with a constant $0.1$dex of scatter, 
and red triangles depict our SHAMacc mock without scatter. 
In all relevant panels of Fig.~\ref{fig:groupfieldscatter}, group galaxies have been 
defined by the richness cut $N_{group}>4.$ 


Evidently, the amount of scatter in SHAM significantly influences the predictions 
for $\Phi_{\mathrm{group}}(L)$ and $\Phi_{\mathrm{field}}(L)$ at the bright end, 
but appears to have little role in the group and field luminosity functions at 
the faint end. A more exhaustive exploration of different prescriptions for 
the scatter may yield a model that correctly predicts the abundance of {\em bright} 
group and field galaxies. Any such model would require significantly larger scatter 
than the fiducial model we have used in this work, which is, in turn, based on the 
success of \citet{trujillo-gomez_etal11} in describing galaxy clustering and 
a variety of other properties using SHAM.  {\bf [ARZ: Can we guess/estimate 
what scatter this would have to be?  0.3dex? 0.4dex?  It would be 
useful to put an approximate number here because I think scatter 
as large as those numbers is too big to be consistent with other papers?]}
However, the robustness of our results to differences at the faint end of the 
luminosity functions indicates that this is a generic weakness of SHAM that cannot 
be overcome by adding scatter alone. We discuss this point further in \S~\ref{section:discussion}.


In the bottom panels of Fig.~\ref{fig:groupfieldscatter} we illustrate how the 
errors in the SHAM prediction for $\Phi_{\mathrm{group}}(L)$ translate into errors 
in the luminosities of the brightest group members (bottom left panel) and 
next-brightest group members (bottom right panel). Qualitatively, the trends 
in the bottom panels reflect the sense of the errors on $\Phi_{\mathrm{group}}(L)$ 
shown in the upper left panel. SHAM predicts brightest group galaxies and 
next-brightest group galaxies that are significantly too bright on average, 
compared to observations.  Moreover, this conclusion is insensitive to scatter, 
so it is, again, a {\em generic weakness of the SHAM assignments}.  


Before proceeding, we note that we include the bottom panels in 
Fig.~\ref{fig:groupfieldscatter} in large part because our results in 
the following sections focus on the relative brightnesses of the 
brightest and next-brightest group members, and so the errors illustrated 
in the bottom panels are germane to all of our results pertaining to the 
SHAM prediction for this relative brightness.


%---------------------------------------------------------------------------------------------------
\begin{figure*}
\centering
\includegraphics[width=8.0cm]{FIGS/groupLF_scatter_compare_0820.eps}
\includegraphics[width=8.0cm]{FIGS/fieldLF_scatter_compare_0820.eps}
\includegraphics[width=8.0cm]{FIGS/BGLF_scatter_compare_0820.eps}
\includegraphics[width=8.0cm]{FIGS/NBGLF_scatter_compare_0820.eps}
\caption{
Fractional difference between a variety of conditioned luminosity functions seen in 
SDSS and the predictions for SHAMacc with our fiducial scatter model (blue diamonds), 
our alternate scatter model (magenta squares), and no scatter (red triangles). 
For this purpose, field galaxies are those galaxies that do not reside in groups, 
and the group galaxy sample is defined by requiring that each galaxy in the sample reside 
in a group with $N>4$ members. The top two panels show the differences in the group (left) 
and field (right) luminosity functions, similar to Fig.~\ref{fig:groupfield}, in order to 
address the influence of scatter in the SHAMacc predictions.  
After rank-ordering the members of each group by their 
brightnesses, we have measured the luminosity function of the brightest groups members as 
well as that of the next-brightest group members. Fractional differences between the SHAMacc 
predictions for the luminosity function brightest galaxies in $N>4$ groups and the observed 
brightest galaxy luminosity function appear in the bottom left panel.  Analogous 
differences for the next brightest group galaxies in the bottom right panel.
}
\label{fig:groupfieldscatter}
\end{figure*}
%-----------------------------------------------------------------------------------------------------


%---------------------------
\subsection{Magnitude Gap Abundance}
\label{subsection:pgap}
%---------------------------

{\bf [ARZ: There is a lot of interpretive work here, which makes this a strong paper.  
However, to some degree, simply ennumerating various quantities like the fossil 
fraction and the magnitude gap abundance function could be very interesting to many 
people.  For each statistic, you should state why your presentation is unique.  
My understanding is that you fossil fraction and magnitude gap abundance functions 
supersede what has been done.  You should state this explicitly where appropriate 
in this section.]}


In \S~\ref{fig:groupfield}, we compared the luminosity functions 
of groups predicted by SHAM to SDSS group data.  In this section, we 
explore a group statistic that is related to the group luminosity function 
known as the ``magnitude gap.''  We define the magnitude gap to be 
the difference in r-band absolute magnitude between the two brightest non-fiber collided 
members of any group, $\monetwo = M_{r,2}-M_{r,1}$, where 
$M_{r,\mathrm{i}}$ is the r-band absolute magnitude of the $\mathrm{i}^{th}$ brightest 
group member. In particular, we will be interested in $\Phi(\monetwo),$ the 
statistical distribution of magnitude gaps, defined so that $\Phi(\monetwo)\dd\monetwo$ 
represents the number density of galaxy groups with magnitude gap $\monetwo$ in a bin of width $\dd\monetwo.$


As we discussed in \S~\ref{section:introduction}, the magnitude gap abundance 
statistic has received significant attention in recent literature.  
Of course, it is possible to enumerate the 
absolute magnitudes of all group members, but the magnitude 
gap has received particular attention for a number of reasons.  
These include: 
(1) the simplicity of a single statistic; 
(2) dynamical friction timescales 
vary in inverse proportion to galaxy mass, so the largest satellites 
merge the most quickly, making magnitude gap a possible indicator 
of the dynamical age of a group; 
and (3) the magnitude gap may help 
to refine mass estimates for optically-identified clusters \citep{hearin_etal12,more12}.  


In this section we focus on the magnitude gap as a simple diagnostic of 
the partitioning of galaxies among groups.  Magnitude gaps provide a test of 
the relationship between galaxies and dark matter halos that is distinct from 
the tests provided by measurements of group abundance (\S~\ref{subsection:gn}) 
and the group luminosity function (\S~\ref{subsection:groupfield}). Galaxy luminosities 
can be partitioned among groups and group members in an infinite variety of ways that 
all lead to the same average luminosity function. An empirical model such 
as SHAM that produces the correct average group luminosity function at a particular 
richness (or range of richnesses) may nevertheless fail to produce the correct distribution of magnitude gaps. 


The abundance of groups by magnitude gap is a rapidly declining function of the gap, 
with approximately 90\% of all groups with $N \geq 3$ members in the 
Mr19 sample having a magnitude gap smaller than $\monetwo = 1.5.$ However, gap abundance 
depends sensitively on group richness, as demonstrated in the top left panel of 
Figure~\ref{fig:pgap}, where we show a histogram (normalized to have unit area) 
of $\Phi(\monetwo)$ exhibited by galaxy 
groups in two different richness ranges. 
The blue histogram traces $\Phi(\monetwo|N=3),$ the gap abundance of groups with $N=3$ members, 
while the red histogram traces $\Phi(\monetwo|9<N<16).$ A comparison of the two histograms 
provides a demonstration of the richness-dependence of $\monetwo$ abundances: richer groups 
tend to have smaller magnitude gaps. This trend has been demonstrated previously in the literature 
\cite[e.g.,][]{paranjape_sheth11,donghia_etal05,tavasoli_etal11}. 

%---------------------------------------------------------------------------------------------------
\begin{figure*}
\centering
\includegraphics[width=8.0cm]{FIGS/gap_richness_bins_0815.eps}
\includegraphics[width=8.0cm]{FIGS/fossil_fraction_vs_N_0815.eps}
\includegraphics[width=8.0cm]{FIGS/gap_abundance_0815.eps}
\includegraphics[width=8.0cm]{FIGS/gap_abundance_residual_0815.eps}
\caption{
{\bf [ARZ: Be a bit careful here.  We may want to have a figure that prints 
clearly in Black \& White just in case.  Also, you should use the notation 
$F_{\mathrm{fos}}$ since you gone through the trouble of defining it in the text.]}
Fossil group statistics.  
In the top left panel the blue (red) histogram traces the relative abundance of groups as 
a function of magnitude gap, $\Phi(\monetwo)$ (normalized to have unit area), exhibited by 
groups with richness $N=3$ ($9<N<16$). The blue diamonds and red triangles 
trace $\Phi(\monetwo)$ of the corresponding Monte Carlo randomizations of the groups. 
{\bf [ARZ: I think the x-axis label should probably just be $N$ rather than $>N$, 
do you agree?]}
In the top right panel we plot the {\em fossil fraction} $\ffos(>N)$, defined as fraction of 
systems in a group sample with magnitude gap $\monetwo\geq 2.$ The richness-threshold defining 
the samples used in the fossil fraction measurements appears on the horizontal axis. 
Blue diamonds (red triangles) show the fossil fraction of the observed (Bolshoi with SHAMacc) 
group samples for a range of richness cuts. In the bottom left panel we plot the magnitude gap 
abundance $\Phi(\monetwo)$, exhibited by groups with $N>4$ members. The observed $\Phi(\monetwo)$ is 
illustrated with blue diamonds; we have chosen a random subsample of our mock groups with a multiplicity 
function matching that of the observed groups and plotted the gap abundance exhibited by this 
random subsample with red triangles. The purpose of the multiplicity function matching is 
to decouple the dependence of the predicted gap abundance from the richness-dependence 
illustrated in the top panels. In the bottom right panel we show the fractional difference 
between the observed gap abundance and that predicted by SHAM models for different amounts of 
scatter between $\mr$ and $\vl$, demonstrating the potential to use $\Phi(\monetwo)$ 
measurements to constrain the scatter in SHAM models.
}
\label{fig:pgap}
\end{figure*}
%-----------------------------------------------------------------------------------------------------

A simple way to gain insight into the sense of this trend is 
to consider a toy model universe in which galaxy groups are 
assembled by randomly drawing galaxy luminosities from a global 
luminosity function $\phiglo(L)$ (for example, a Schechter function).  As the richness $N$ of the toy groups 
increases, the number of random draws from $\phiglo$ increases, and the probability 
that a very bright member is drawn increases. Denoting the luminosity of the $\ith$ brightest 
members as $L_i,$ the expectation value of $L_i$ becomes brighter with increasing $N.$ As the 
number of random draws increases, the expectation value of $L_1$ is the first to become brighter 
than $L_*,$ the exponential cutoff of $\phiglo,$ and when this occurs there is a rapid decrease 
in the rate at which the expectation value of $L_1$ brightens with increasing $N.$  The 
reason for this rapid decrease is because of the exponential suppression in the luminosity 
function for galaxies with luminosties exceeding $L_*$; most draws correspond to lower luminosities.  
A relatively larger number of draws is required for the expectation value of $L_2$ 
to exceed $L_*,$ and so as $N$ continues to increase the expectation value of the 
ratio $L_1/L_2$ decreases.\footnote{Note that this intuitive explanation also 
demonstrates that the root cause of the shrinking of the magnitude gap with $N$ 
random draws is that the slope of $\phiglo$ steepens as the brightness increases. 
If $\phiglo$ did not have this property, for example if the global luminosity function 
were a simple power law, $\phiglo(L)\propto L^{\alpha},$ then the expectation 
value of the magnitude gap in the random draw universe would be entirely independent 
of $N$ since the expectation value of $L_1$ and $L_2$ would each brighten with 
increasing $N$ at the same rate.}


To illustrate this point explicitly, the blue diamonds and red triangles 
in Figure~\ref{fig:pgap} show histograms of $\Phi(\monetwo)$ in Monte Carlo (MC) 
realizations of this toy universe. For each observed Mr19 group of richness $N$, 
we have constructed $1000$ realizations of the group by randomly drawing $N$ times 
from the {\em observed} $\Phi(L|N\geq3)$ (note that the observed luminosity function 
is used and not a Schechter function approximation). 
Thus, for group samples in each richness bin plotted in the top, 
left panel of Fig.~\ref{fig:pgap}, the multiplicity functions of the observed 
and MC groups match {\em exactly}. The difference between the r-band magnitudes 
of the brightest two draws gives the $\monetwo$ value of the MC group. 
Blue diamonds give the $\Phi(\monetwo|N=3)$ that results from this 
exercise, red triangles represent $\Phi(\monetwo|9<N<16).$ 
Of course, this toy model ignores any evolution of, or interactions between, 
group members during the process of group formation, although the broad similarity 
between $\Phi(\monetwo|N)$ in the MC and in Mr19 data demonstrate 
that this model nonetheless provides a reasonable approximation of the relationship 
between richness and magnitude gap.


Our chief goal in this section is to demonstrate the utility of the 
observed gap abundance $\Phi(\monetwo)$ for constraining the galaxy-dark matter connection, 
with a particular focus on SHAM-based models. A basic consequence of the 
relationship between gap and richness is that the multiplicity function $g(N)$ 
plays a critical role in $\phi(\monetwo).$ 
As shown in \S~\ref{subsection:gn}, SHAMpeak and SHAM0 models 
over- and under-predict  $g(N)$ at moderate and large values of group richness, respectively.  
Therefore, we should expect that these models will not predict the correct gap abundance function, 
$\Phi(\monetwo).$ However, it is still useful to explore the distribution of 
magnitude gaps, given a common richness or distribution of richness.  Such 
a statistic eliminates the multiplicity function $g(N)$ as a possible cause 
of discrepancy, so it can serve as a pure test of the ability of the SHAM formalism 
to allocate the brightest galaxies into physically associated, group-sized systems.  
We proceed to undertake such a comparison by randomly selecting a subsample of 
mock groups from our SHAM catalogs with a multiplicity function that matches that 
of the observed, SDSS Mr19 sample.  Specifically, we compare the observed number density 
of groups as a function of magnitude gap, $\Phi_{\mathrm{SDSS}}(\monetwo)$ to that 
in a subsample of our SHAM mocks restricted to have an identical group multiplicity 
function, $\Phi_{\mathrm{mock}}(\monetwo|g=g_{\mathrm{SDSS}})$.  


The top right panel of Figure~\ref{fig:pgap} shows one result of such a 
test. Plotted as blue diamonds in Fig.~\ref{fig:pgap} is the {\em richness-threshold 
conditioned fossil fraction} $\ffos(>N),$ defined as the fractional abundance of 
SDSS Mr19 galaxy groups with more than $N$ members that have $\monetwo\geq2,$: 
%
\begin{equation}
\ffos(>N)\equiv\frac{\int_{2}^{\infty}\dd\monetwo\Phi(\monetwo|>N)}{\int_{0}^{\infty}\dd\monetwo\Phi(\monetwo|>N)}.
\end{equation}
%
The fossil fraction is a measure of the size of the large-gap tail of the 
magnitude gap distribution; 
galaxy groups with $\monetwo \geq 2$ whose X-ray brightnesses are 
greater than some threshold value (commonly $L_{X,bol}>10^{42}\mathrm{erg/s}$) 
are often referred to as {\em fossil groups}, and are conventionally thought 
to be galaxy systems that assembled most of their mass at high redshift, 
representing the end products of galaxy group evolution. 
We address the consistency of this interpretation with our findings 
in \S~\ref{section:mcs} and \S~\ref{section:discussion}. 
The decrease of $\ffos(>N)$ with increasing $N$ reflects the relationship 
between richness and gap discussed above: large gap systems are rarer in 
systems of larger richness.


Plotted in red triangles in the top right panel of Figure~\ref{fig:pgap} 
is the SHAMacc prediction for $\ffos(>N)$ from our fiducial mock catalog after 
selecting a random subsample with a $g(N)$ distribution that matches the observed 
Mr19 multiplicity function. For both the mock and observed data points the error bars 
come from bootstrap resampling. In the case of the SHAMacc groups, each bootstrap 
realization corresponds to a new, random selection of a subsample of the mock groups 
with a matched $g(N)$ distribution. Regardless of the richness threshold, the difference 
between the observed fossil fraction and that predicted our fiducial SHAMacc model is less than 
$2\sigma$. {\bf [ARZ: This isn't entirely obvious from the figure: do you mean less 
than 2-sigma at any given point, N, or less than 2-sigma even after summing over 
all points.]}


{\bf [ARZ: Is this the most complete exposition of the fossil group abundance 
and the magnitude gap distribution?  If so, you should say that somewhere 
and make this result known.]}
In the bottom left panel of Fig.~\ref{fig:pgap}, we show the gap abundance $\Phi(\monetwo)$ predicted by 
our fiducial mock catalog and compare it to the observed Mr19 abundance. 
As in the upper right panel of Fig.~\ref{fig:pgap}, the subsample of mock groups 
has been chosen to match the observed multiplicity function. Thus this figure illustrates 
the results of a direct test of the abundance matching prediction for the relative 
brightnesses of the brightest group galaxies, independent the observed group multiplicity. 
There is less than a $1\sigma$ difference between the observed $\Phi(\monetwo)$ and our 
fiducial prediction, which constitutes a new success of SHAM-based models for the 
galaxy-dark matter connection. With the same model of scatter, the matched $g(N)$ 
prediction for $\Phi(\monetwo)$ when abundance matching with either $\vpeak$ or $\vzero$ 
results in less than a $2\sigma$ discrepancy with the data.  The magnitude 
gap is not an effective statistic with which to discriminate between the 
different SHAM algorithms in common use.  
\footnote{Note that if this comparison is done without first matching the 
predicted $g(N)$ to the observed group multiplicity then SHAMpeak and SHAM0 
predictions for $\Phi(\monetwo)$ are in stark disagreement with the data, 
demonstrating the importance of multiplicity matching when one is interested 
solely in the relative luminosities of the brightest group galaxies.}   


The success with which SHAM describes the distribution of magnitude gaps 
may seem less interesting because we have already shown that SHAM does not 
accurately predict the luminosities of our group galaxies 
(\S~\ref{subsection:groupfield}, Figs.~\ref{fig:groupfield} \& \ref{fig:groupfieldscatter}). 
However, we first note that although these results are related, the gap abundance prediction 
does not follow directly from the group galaxy luminosity function. For example, 
one could have imagined the average luminosities of group galaxies to have 
been {\em correctly} predicted, while the distribution of magnitude gaps was 
incorrect due to a failure of SHAM to correctly arrange bright galaxies within groups. 
In this way, one can see that magnitude gaps test not just the mean conditional 
luminosity function (CLF) of group galaxies, but also correlations between the luminosities 
of group members.

Additionally, magnitude gap abundance observations provide constraints on 
the SHAM models that are {\em complementary} to measurements of group multiplicity. 
We illustrate this in the bottom right panel of Fig.~\ref{fig:pgap}.  
In that panel, we show the fractional difference between the observed 
abundance of groups as a function of magnitude gap and that predicted by 
SHAM.  The difference, $\Delta\Phi(\monetwo) = \Phi_{\mathrm{mock}}(\monetwo)-\Phi_{\mathrm{SDSS}}(\monetwo)$, 
analogous to the group and field luminosity functions defined  in \S~\ref{subsection:groupfield}.  
The red triangles show results from our fiducial scatter model 
($0.2$dex of scatter at the faint end, $0.15$dex at the bright end), 
the magenta squares depict our alternate scatter model (constant scatter of $0.1$dex), 
and the blue crosses show SHAMacc with no scatter. 
It is evident that $\Phi(\monetwo)$ is a relatively 
sensitive probe of the underlying scatter between luminosity and $\vl$.
The alternate scatter model is discrepant with the data at $2.7\sigma$, 
the no scatter model at $4.1\sigma,$ strongly suggesting that $\Phi(\monetwo)$ observations 
can be exploited to constrain the scatter between luminosity and halo circular velocity. 
We reiterate that the gap abundance prediction has been decoupled from the group multiplicity 
prediction for each model, so the complementarity of the constraints on the SHAM model 
provided by $\Phi(\monetwo)$ and $g(N)$ can be realized as these statistics can be used 
concurrently. Magnitude gap or related statistics may thus prove to be incisive statistics 
with which to constrain scatter in galaxy-halo assignments, but we relegate 
a detailed study of this possibility to future work.


We repeated this entire exercise for mock groups identified in real space 
(as opposed to redshift space), and found that the observed $\Phi(\monetwo)$ and 
that predicted by our fiducial mock are different at a level of $\simeq 4.5\sigma$. 
This demonstrates the importance of redshift-space group-finding in making the 
prediction for the gap abundance. 
To our knowledge, we have performed this analysis for the first time; 
all previous studies relying on numerical simulations to predict gap abundances 
have used ``halo-level'' abundances as the prediction, 
in which halo membership is used to define group membership. 
However, real-space predictions systematically under-estimate 
the abundance of low-gap systems, a fact that we find to hold true 
regardless of the SHAM prescription. This is sensible since interlopers 
occur more often in redshift-space groups, and interlopers can only reduce 
the gap, so it is natural to expect that including interlopers by 
finding the groups in redshift-space should boost the low-gap abundance.  
Any theoretical study of the magnitude gap abundance must properly account 
for interlopers due to redshift-space projection effects in order to predict $\Phi(\monetwo)$ correctly. 
 

%---------------------------
\section{Data Randomization Comparisons}
\label{section:mcs}
%---------------------------

In the previous section, we explored a variety of properties of 
galaxy groups including group multiplicity and magnitude gap.  
As we pointed out in our discussion of Figure~\ref{fig:pgap} at the beginning in \S~\ref{subsection:pgap}, 
it is natural to expect the number of group members to be correlated with, for example, 
the luminosity of the brightest group galaxy or the magnitude gap. 
The reason is simple. Consider a toy model in which the luminosities of 
group members are consistent with being random draws from a universal group luminosity function. 
In this case, the typical luminosity of the brightest group galaxy will increase with 
the number of group members.  Likewise, the magnitude gap will decrease with 
the number of group members. The fidelity with which such a toy model represents 
the real universe has been explored previously by 
\citet{paranjape_sheth11}, and it is indeed interesting to determine whether 
observed galaxies are consistent with such a simple hypothesis.  

We revisit this investigation here and proceed as follows. For a given set of groups, let 
$\Phi(\monetwo|> N)$ be the abundance of the groups of 
a given magnitude gap subject to the condition that 
the group has greater than $N$ members (of course, other conditions 
could be placed on this distribution as well, 
see below for further discussion).  Assume 
some luminosity function of galaxies within such 
groups, $\Phi(L|> N)$.  If we assume that galaxy 
luminosities are consistent with random draws from 
$\Phi(L|> N)$, there is a definite prediction for 
the distribution of magnitude gaps in this random-draw 
hypothesis, $\phirand(\monetwo|> N)$.  Of course, 
real data need not be consistent with this hypothesis, 
so it is interesting to examine deviations from the 
random draw hypothesis.  We do this by defining the 
fractional deviations from the random-draw prediction, 
%
\beq
\label{eq:adf} 
\Psi(\monetwo|> N) \equiv 
\frac{\Phi(\monetwo|> N)-\phirand(\monetwo|> N)}{\phirand(\monetwo|> N)}.
\eeq 
%
Of course, one can make different assumptions about the luminosity 
function from which the galaxies are being drawn in order 
to test different hypotheses. For example, one could draw 
the luminosities from the all-galaxy luminosity function 
$\Phi(L),$ rather than a richness threshold-conditioned LF. 
We intend to explore these and related details in a future follow-up paper.

{\bf [ARZ: Isn't ``$>N=3$'' a goofy notation?  Shouldn't it be 
``$>3$'' to follow the prototype given in the paragraph above?  
After all, $N$ is a random variable and 3 is a particular value 
of that variable that may or may not be realized.
I think this notation should be changed a bit to reflect this 
or something similar to this.  Also, the description here 
seems to have slightly different numbers from the figure 
labels.  Please check this for consistent notation.]}
With the blue diamonds in Figure~\ref{fig:psigap}, we show 
$\Psi(\monetwo|>N=3)$ for SDSS Mr19 groups in the 
top panel and  $\Psi(\monetwo|> N=10)$ in the bottom panel.  
This is similar to the case that 
was tested in \citet{paranjape_sheth11}, except those authors 
used an analytical fit \citep{bernardi_etal10} to $\Phi(L|> N),$ 
and restricted attention to $N>10.$
The $\Psi(\monetwo|> N)$ in Fig.~\ref{fig:psigap} are 
clearly inconsistent with zero.  The statistical significance 
of this difference is $\simeq 4.6\sigma$ for $N > 3$ groups 
and $\simeq 2.7\sigma$ for $N > 10$ groups.  Alternatively, 
the p-values from two-sided KS tests are $\lesssim 10^{-4}$ 
in both cases.  This test yields the clear conclusion that 
the luminosities of galaxies within SDSS groups are not 
consistent with random draws from the global luminosity 
function.  This is a direct contradiction of the conclusions drawn 
in \citet{paranjape_sheth11}. The source of the discrepancy between 
these two conclusions lies in the treatment of fiber collisions, 
which we provide a detailed account of in Appendix B.

We have also measured the $\Psi(\monetwo|> N)$ that results from 
an alternative data randomization procedure, which we describe as follows. 
First, we divide our SDSS Mr19 sample of galaxies into ``centrals'' and ``satellites''; 
the set of centrals is defined to be those galaxies that are the brightest among 
the galaxies in the group of which they are a member; the set of satellites is 
the complement to the set of centrals. For each group of richness $N$ in the 
sample being randomized, we construct $1000$ realizations of the group. For each 
realization, we fix $L_1,$ the luminosity of the ``first'' randomized group member, 
to be equal to $L_{cen},$ the luminosity of the central galaxy in the group that 
is being randomized. The luminosities of the remaining $N-1$ members are drawn 
at random from $\Phi_{sat}(L|L_{cen}),$ the luminosity function of all satellite 
galaxies that are found in groups whose central galaxy has a luminosity within $0.2$dex of $L_{cen}.$

This randomization scheme may appear unfamiliar at first glance, but in fact 
it is well-motivated by and quite similar to the method by which mock catalogs of 
galaxies are constructed from N-body simulations in the standard Conditional Luminosity 
function (CLF) formalism. For CLF-based mocks, the value $L_{cen}$ is chosen at random from 
a log-normal luminosity function whose mean and spread are governed by the mass of the 
dark matter halo, and the luminosities of the remaining galaxies associated with the halo 
are chosen from a modified Schechter function whose form is governed by $L_{cen}.$ Our approach 
to the construction of our randomized groups is very similar except that we know the richnesses 
of the groups rather than their masses. By keeping fixed the luminosity of the central galaxy 
in each randomized group, this randomization preserves the relationship between $L_{cen}$ and $N$ 
exhibited by the observed groups. Rather than using an analytical fit to $\Phi_{sat},$ we 
use the data itself determine this distribution; our choice to condition the luminosity 
function from which the brightness of each randomized group's satellites are drawn 
mirrors the convention adopted by the standard CLF formalism. 

A non-negligible fraction of the randomly-drawn satellites are brighter than 
the central galaxy in the randomized group. However, in calculating the magnitude 
gap of each randomized group we proceed exactly as we do with the data and 
define $\monetwo$ to be $M_{r,1}-M_{r,2},$ the magnitude difference between the 
brightest two members. The resulting $\Psi(\monetwo|> N)$
is plotted with red squares in both panels of Fig.~\ref{fig:psigap}. The second data 
randomization performs better than the first at faithfully describing the relative 
brightness of the brightest group galaxies in rich groups. For this randomization, 
the observed $\Psi(\monetwo|> N=9)$ differs from zero at a level of just $\simeq 1.7\sigma,$ 
although $\Psi(\monetwo|> N=2)$ is still $\simeq 4.1\sigma$ discrepant with zero. 
Because of the similarity of the second data randomization to the standard CLF formalism, 
these results suggest that the CLF adequately describes the arrangement 
of the brightest galaxies within rich groups, but may be an inadequate description 
of the luminosities of satellite galaxies in very poor groups.

There are several possibilities that may explain the difference of 
$\Psi(\monetwo|> N=2)$ from zero. First, the satellite luminosity function 
in poor groups may require conditioning from an additional variable besides $L_{cen}.$ 
Second, correlated draws from $\Phi_{cen}(L)$ and $\Phi_{sat}(L)$ may be required to describe accurately 
the observed magnitude gap abundance in low-richness groups. 
Third, this could be a manifestation of the difficulty of group identification for systems with 
just three or four group members. We will attempt to discern between these possibilities in a follow-up paper.

{\bf [ARZ: I think it still remains an obvious omission in this context not to 
have some treatment of the randomization tests for the brightest galaxy distribution. 
I think you should consider seriously including that test in this section to make 
is feel more complete and well thought out. Get back to me on your thoughts in 
this regard.]}
While we have focused entirely on magnitude gaps $\monetwo$ in this section, 
it is also interesting to construct analogous statistics from alternative 
observables such as the luminosity function of the brightest group galaxy. 
Studying these and other such statistics constructed from a variety of data 
randomizations may yield new insight into the physics of galaxy group formation. 
We explore this promising technique more fully in a follow-up paper.


%---------------------------------------------------------------------------------------------------
\begin{figure}
\centering
\includegraphics[width=8.0cm]{FIGS/psi_both_Nge3_0826.eps}
\includegraphics[width=8.0cm]{FIGS/psi_both_Nge10_0826.eps}
\caption{
Randomization tests of magnitude-gap statistics. 
In blue diamonds we plot $\Psi(\monetwo|> N)$ (see Eq.~\ref{eq:adf}) observed in SDSS
Mr19 when the data randomization procedure is to draw brightnesses for
all the group members from the global $\Phi(L|> N).$ We show results
for a richness threshold of $N> 2$ in the top panel, and $N> 9$ in
the bottom panel. In red squares we plot $\Psi(\monetwo|> N)$ when the
randomization procedure is to keep fixed the luminosity of the
brightest group member and to randomly draw brightnesses for the
remaining galaxies from $\Phi_{sat}(L|L_{cen}).$ See text for further
details about the data randomizations. Each of the $\Psi(\monetwo|> N)$ 
are distinct from zero, indicating that the distribution of magnitude gaps 
within groups is inconsistent with either of the simple random-draw 
hypotheses examined in these panels. 
}
\label{fig:psigap}
\end{figure}
%-----------------------------------------------------------------------------------------------------

%--------------------------
\section{Discussion}
\label{section:discussion}
%---------------------------

{\bf [ARZ: I think you are a bit too cautious throughout this 
entire discussion section.  If we should make specific statements, 
even if we refer to them as ``tentative'' conclusions.  Otherwise, 
the results won't be noticed.  We have specific, but perhaps tentative, 
statements that can be made about SHAM models and CLF models.  Please 
try to rework this, along with my comments below, in order to have the 
discussion make specific statements.]}

We have used a volume-limited catalog of galaxy groups observed in
SDSS DR7 to provide a number of new tests of the abundance matching
prescription for connecting galaxies to dark matter halos. In
\S~\ref{subsection:gn} we demonstrated that our fiducial mock galaxy
catalog, constructed by abundance matching using $\vacc,$ as the luminosity proxy for subhalos 
accurately reproduces the observed group multiplicity function $g(N),$ that is,
the abundance of groups as a function of group richness. This
constitutes a new success of SHAM that is distinct from previous tests
that rely on measurements of galaxy clustering. Additionally, we
showed that mock catalogs using $\vpeak$ or $\vzero$ as the abundance
matching parameters incorrectly predict group multiplicity
measurements and straddle the $g(N)$ predicted by $\vacc-$based
catalogs (Figure ~\ref{fig:gn}). We have checked that this qualitative
behavior holds true for models with very different (or without)
scatter between luminosity and $\vmax,$ as well as for volume-limited
samples with different brightness thresholds, 
indicating that this a generic conclusion. This is
particularly interesting in the context of a recent study by
\citet{watson_etal11}, who showed that incorporating stellar mass loss
into SHAM-based models of galaxy formation improves the predictions
for galaxy clustering. Since a subhalo's $\vmax$ value at the time of
accretion represents an intermediary stage of dark matter mass loss
between $\vpeak$ and $\vzero,$ our results point towards the
possibility that $g(N)$ measurements may have the potential to
constrain evolution of stellar mass (both new star formation and 
stellar  mass loss) in satellite galaxies within group and cluster halos.

We have developed a novel implementation of SHAM that allows for the
construction of mock galaxy catalogs with a luminosity function
$\Phi(L)$ that exactly matches the $\Phi(L)$ exhibited by an 
observed galaxy sample, even when scatter between halo circular velocity 
and galaxy luminosity is included. We have
exploited this implementation to test the ability of SHAM to predict
the galaxy luminosity function conditioned on whether the galaxies in
the sample are members of groups. 

{\bf [ARZ: The discussion in the next two paragraphs 
needs some work.  It is too vague right now 
and I'm not sure what you want to get at.  There are differences 
which we have become aware of only due to a private communication.  
One thing that we need to know (and perhaps to state here) is how 
closely Reddick et al. get the global LF.  Could it be that 
their SHAM method doesn't match the global LF correctly and that 
this induces other problems?  There are only a small set of possible 
sources for any differences.  One could be the SHAM algorithm.  Our 
method guarantees the correct, global LF.  Theirs doesn't.  Alternatively, 
group identification or some other such difference could be relevant.]}
We find that field (group) galaxies in SHAM-based catalogs are systematically 
too dim (bright), and that this behavior holds true in all of the 
SHAM models we studied. This indicates that none of the popular 
SHAM-based models allocate galaxies to field and group environments 
correctly. 


{\bf NEW PARAGRAPH HERE.}
In a study closely related to ours, \citet{reddick_etal12}
construct SHAM-based mock catalogs of galaxy groups and produced exhaustive 
constraints on SHAM-based models of the galaxy-halo connection, so it 
is useful to compare our results with theirs.   
We find somewhat different trends in $\Delta\Phi_{group}$ and $\Delta\Phi_{field}$ 
with luminosity than we report here.  \citet{reddick_etal12} find 
{\bf ... whatever it is they find. they also find this other 
thing (private communication, Rachel Reddick).}  
Furthermore, there exists a $ \sim 4\sigma$ discrepancy
between \citet{reddick_etal12} best-fit prediction for $\Phi_{field}(L)$ and 
the observed field luminosity function using $\vpeak$ 
(private communication, Rachel Reddick). It is possible that 
this discrepancy could, in part, induce the trends observed above.


There are few additional possibilities that may account for the differences
between our results and those of \citet{reddick_etal12} because 
we analyze the same N-body simulation, using the same halo-finder, 
and compare to the same galaxy sample. The primary differences 
between our methodologies lie in the details of
our SHAM models and in our group-finding algorithms. If either of
these two differences accounts for the different luminosity trends, 
it would have important consequences for the construction of 
mock galaxy catalogs that faithfully represent the
properties of observed galaxies. This emphasizes a need for 
systematic and detailed examinations of the influences of 
mock catalog construction, group finding, and other methodological 
issues in order to understand the potential systematic differences 
induced by these choices.


One of the most common statistics used to quantify magnitude gaps is
the {\em fossil fraction}, defined as the fraction of galaxy groups in
a given sample with a magnitude gap $\monetwo \geq 2$.  This is the
statistic we present in the upper right panel of Figure
\ref{fig:pgap}. We have not adopted an X-ray brightness threshold
criterion on our groups, nor have we imposed a maximum distance from
the group centroid used to restrict the set of group members used in
the measurement of the magnitude gap, and so a direct comparison
between the fossil fraction we measure and the $8-20\%$ reported in
\citet{jones_etal03} would not be meaningful. Among the existing
results in the literature, the approach in \citet{yang_etal08} is most
similar to ours: the authors employ their group-finding algorithm on a
volume-limited, optical sample of galaxies and simply define
$\monetwo$ to be the difference in r-band magnitude between the
brightest two group members. Thus their definition of a fossil is very
similar to ours, and they quote fossil fractions for several different
ranges of group mass, ranging from $0.5\%$ for groups of mass
$\sim10^{14.5}\msun$ to $18-60\%$ for groups of mass
$\sim10^{13}\msun.$ Unfortunately, the mass-binning of these values
makes a direct, quantitative comparison impossible because, unlike the
group-finding algorithm in \citet{yang_etal08}, our algorithm does not
enforce the same assumptions about dark matter halo properties, 
and so groups found with our algorithm are not
inextricably connected with a unique prediction for group
mass. Nonetheless, the fossil fractions quoted in \citet{yang_etal08}
do appear to be significantly higher than those we report in
Fig.~\ref{fig:pgap}, which may be another sign of the sensitivity of
group properties to the algorithms with which they are found. 


Recently, \citet{proctor_etal11} claimed that the low richness of the
ten fossil groups they studied indicates a problem for the standard
scenario of fossil group formation. In particular, \citet{proctor_etal11} 
argued that the fact that their fossil groups are under-rich at all 
observed luminosities is difficult to understand within the standard 
$\lcdm$ theory of structure formation.  Our results are a direct
refutation of this claim. The magnitude gap abundance and fossil
fraction exhibited by our fiducial mock catalog {\em constitutes} the
simplest $\lcdm$ prediction for these statistics, and we have shown that 
these predictions are in good agreement with the data. In particular, 
we find, both in our mock catalogs and in the observed SDSS DR7 groups, 
that groups with large magnitude gaps are under-rich at all luminosities. 
We thus find no evidence that the abundance or properties of fossil systems 
presents a problem for the $\lcdm$ picture of structure formation.


Data randomization techniques similar to the ones we use in
\S~\ref{section:mcs} have been used previously in the literature. For
example, \citet{jones_etal03}, \citet{dariush_etal07}, and
\citet{tavasoli_etal11} all addressed the connection between richness
and magnitude gap with Monte Carlo (MC) realizations of a group
sample. These authors constructed a population of $10^{4}-10^{6}$
Monte Carlo groups by drawing a fixed number of times from a global
Schechter luminosity function to populate each MC group with a set of
galaxies. In finding that the fraction of their MC groups with
$\monetwo \geq 2$ was lower than that of the groups in their sample,
they each concluded that fossil groups do not 
have a ``statistical origin.''  The authors interpreted 
these exercises as evidence that fossil groups do not arise 
as extreme realizations of a Poisson process based on the global 
galaxy luminosity function, but through a 
dynamical process that preferentially eliminates satellite 
galaxies with large luminosities, namely mergers driven 
by dynamical friction.


More recently, \citet{paranjape_sheth11} generalized this technique to
construct a Monte Carlo realization of a group sample with the same
multiplicity function as the observed group sample, finding that
$\Phi(\monetwo),$ the abundance of groups as a function of magnitude
gap, is well-described by their Monte Carlo population. As discussed
in \S~\ref{section:mcs} and in Appendix B, we repeat the
\citet{paranjape_sheth11} analysis with an improved treatment of fiber
collisions and reach the opposite conclusion.\footnote{We note that
when we run our analysis pipeline on their data set and adopt their
fiber collision convention, we recover their results in full
quantitative detail.} {\bf [ARZ: Do we actually disagree that 
such comparisons can be used as evidence of a dynamical origin? 
Part of the problem with making such a statement is that 
'dynamical origin' can mean so many things. It could even 
mean simply that different things happen in groups from 
in the field.   
I would say that if we falsify the random-draw hypothesis, 
then it is false.  Galaxy groups do not get their galaxies 
assigned luminosities as a result of random draws from the global 
luminosity funtion.  I have commented out a few sentences here 
in order to rephrase this the way I would say it.]}
%
%However, we disagree that such comparisons can
%be used as evidence for a dynamical origin of the magnitude gap (note
%that \citet{paranjape_sheth11} offered no such interpretation). To
%support a claim that magnitude gaps in galaxy groups have a dynamical
%origin, one must first construct a proposed theory in which gaps form
%via dynamical processes, and then test this theory's predictions
%against observations. 
%

We have extended such comparisons by constructing the predictions 
for group properties in a simple, but specific, theory and tested 
the predictions of the theory.  The cosmological simulation on which our mock
catalog is based traces the evolution of a $\lcdm$ universe from the
initial seeds of structure formation through to the present day,
including the dynamical processes conventionally thought to
determine the magnitude gap. The successful prediction for
$\Phi(\monetwo)$ of our fiducial mock catalog thus provides strong
supporting evidence that the magnitude gap exhibited by galaxy groups
is influenced by dynamical processes. On the other hand, tests based
on the Monte Carlo realizations described above can only determine
whether or not knowledge of the richness of groups together with a
universal galaxy luminosity function provides sufficient information
to predict the magnitude gap distribution. In \S~\ref{section:mcs} we
established that such information is insufficient. However, from this
we can only conclude that knowledge of some other group property
besides richness is required to predict the observed $\Phi(\monetwo)$. 
This insufficiency does not reveal the origin of
systems with a large gap.


{\bf [ARZ: The use of this as a ``test'' of the CLF formalism 
here is not sufficiently well developed for most readers to 
understand it and needs to be reworked to be clear, and explicit. 
I think you have to state this very explicitly, for example...  
The CLF formalism assumes x, y, z.  We construct data randomizations 
in which the central galaxy luminosity is fixed and the satellite 
luminosity distribution is determined only by the central luminosity. 
This design mimics the CLF formalism.  However, notice that it is 
more general. We have not specified the form of the CLF.  We are 
allowing the data to determine whether or not the underlying premises
of the standard CLF formalism, namely that the satellite luminosities can 
be drawn from a distribution conditioned on the central galaxy luminosity, 
can be supported in detail by existing data. ... After some sort of exposition 
along those lines, you can continue.  Play with this to make some 
clear, explicit statements.]}  
We have generalized these data randomization techniques to
test an alternative hypothesis for the arrangement of the brightest
galaxies into groups, testing the premises behind the CLF.
These test demonstrate through explicit examples that randomization 
techniques can be quite useful despite our words of caution in the 
preceding paragraph. We have tested for the possibility that
draws from $\Phi_{sat}(L)$ and $\Phi_{cen}(L)$ are correlated (Fig.~\ref{}. 
We find no evidence for such correlations when restricting our group sample to
those which have $N \geq 10$ members.  
However, when we include all groups
with $N \geq 3$, members the data is poorly described by its
randomization. The group multiplicity function is a 
rapidly-declining function of the number of group members, 
$g(N)\propto N^{-2.5},$ so systems with fewer
members dominate the $\Phi(\monetwo)$ measurement.  
Accordingly, the measurement on the higher richness threshold 
sample is subject to larger errors, though it is possible that 
this result could indicate that the luminosities of 
satellite and central galaxies are more strongly correlated in poor 
groups than they are in rich groups. {\bf [ARZ: Although, if you 
look at the figure, probably not.  It's just that for the 
poorer groups you have smaller error bars. So I added the 
last sentence above.]}

The fact that $\psi(\monetwo | >2) \neq 0$ indicates that 
the distribution of satellite luminosities, 
$\Phi_{sat}(L)$ cannot be determined solely by 
the luminosity of the central galaxy. Additional 
information about the group is necessary in order to 
fix the distribution of satellite luminosities.  This result 
contradicts the premise of the CLF (and notice we 
did not assume a functional form in order to arrive 
at this contradiction). Standard CLF models, which do 
not contain any such additional information, have been shown to reproduce 
a wide range of astronomical data.  The gap abundance provides a {\em new} 
statistic with which to test galaxy-halo assignment models such as the CLF, 
and so it would be very interesting to determine what modifications 
to the standard CLF formalism are required
in order to predict $\Phi(\monetwo)$ accurately. 
Yet, before such a result could be established unequivocally, 
it would first need to be confirmed that
$\Psi(\monetwo|>N=2) \neq 0$ generally, {\em irrespective} of the 
details of the group-finding algorithm. We leave the exhaustive 
study of this issue as a task for future work. 


%--------------------------
\section{Concluding Remarks}
\label{section:conclusion}
%---------------------------

We have used the SDSS Mr19 catalog of galaxy groups to provide a
series of new tests of models for the connection between galaxies and
dark matter halos. We conclude this paper with a brief summary of our
primary results.

\ben
\item We have developed a novel implementation of SHAM that allows for
the rapid construction of a mock galaxy catalog with a brightness
distribution that {\em exactly matches} any desired luminosity
function, {\em even after scatter has been included.} 
\item Our fiducial SHAM model, based on abundance matching on $\vacc$
with $0.2$dex of scatter at the faint end and $0.15$dex at the bright
end, accurately predicts group multiplicity, the abundance of groups
as a function of richness, $g(N),$ a new success for the abundance
matching prescription.
\item The $g(N)$ predictions based on SHAM models using $\vpeak$
and $\vzero$ do not match the observed group multiplicity function.  
In fact, these predictions straddle the $\vacc$ prediction, 
so measurements of group multiplicity may provide 
a promising avenue for constraining models of
satellite mass stripping.
\item The group galaxy luminosity function $\Phi_{group}(L)$ and field
galaxy luminosity function $\Phi_{field}(L)$ are predicted rather
poorly by our mock catalogs, with SHAM group galaxies being
systematically too bright and SHAM field galaxies systematically too
dim. Since our all-galaxy luminosity function exactly matches that of
the observed catalog by construction, this shortcoming must be due to
an erroneous allocation of galaxies into group and field environments. 
We find this to be true in all of the variations of SHAM
catalogs that we explored, suggesting that this is a generic
weakness of the SHAM prescription.
\item Our fiducial SHAM model, as well as models using $\vpeak$ and
$\vzero$ with the same amount of scatter, accurately predicts the
observed abundance of groups as a function of magnitude gap,
$\Phi(\monetwo),$ suggesting that the prediction for the relative 
brightnesses of galaxies in groups is a new success of the SHAM
paradigm.
\item The gap abundance prediction is quite sensitive to the amount of
scatter between luminosity and $\vmax,$ suggesting that
$\Phi(\monetwo)$ measurements may be a new way to constrain the
scatter in abundance matching.
\item The observed gap abundance is inconsistent with the hypothesis
that the gap is determined by a set of random draws from a universal
luminosity function, contradicting recent results of {\bf XXXXXXXXXXXXX}.
\item The hypothesis that satellite galaxy brightnesses are drawn at
random from $\Phi_{sat}(L|L_{cen})$ is inconsistent with the SDSS DR7 
groups, at least for relatively low-richness groups. This contradicts 
one of the premises of the CLF formalism and indicates that draws from
$\Phi_{sat}(L)$ and $\Phi_{cen}(L)$ are correlated in small groups, a
possibility that we will pursue in future work.  
\een


\section*{acknowledgments}

We have benefitted greatly from discussions with 
Nick Battaglia, Surhud More, Maya Newman, and ...
APH and ARZ were supported in part by the Pittsburgh 
Particle physics, Astrophysics, and Cosmology Center (PITT PACC)
at the University of Pittsburgh. 
APH ... {your fermilab source of support goes here}.
The work of ARZ is also 
supported in part by the National Science Foundation 
through grant AST 1108802.  AAB ... JAN...



\bibliography{pgap}


%--------------------------
\section*{Appendix A}
\label{section:appendix}
%---------------------------


In this Appendix we give a detailed account of our implementation of
the abundance matching procedure to assign galaxies with r-band
luminosities to dark matter halos. As discussed in
\S~\ref{section:mocks}, a map from the maximum circular velocity of a
halo, $\vmax,$ to an r-band luminosity $\mr$ is provided by the
implicit relation given by Eq.~\ref{eq:lv}. As we demonstrate in \S~\ref{section:prediction}, different choices for the abundance matching parameter (that is, $\vzero,$ $\vacc,$ $\vpeak$) result in mock galaxy catalogs with different properties, and so this choice has important consequences in the modeling of the galaxy-halo connection. We remind the reader that we denote mock catalogs constructed by abundance matching with $\vzero$ as ``SHAM0", with $\vacc$ as ``SHAMacc", and with $\vpeak$ as ``SHAMpeak", and that the mock catalog referred to in the text as our fiducial model is a SHAMacc catalog. Since the novel features of our SHAM implementation method are the same regardless of this choice, throughout this Appendix we simply refer to the abundance matching parameter as $\vl.$

%We define the circular speed used in our luminosity assignment to be $\vl = \vl$ for distinct halos and $\vl = \vacc$ for subhalos, where $\vacc$ is defined to be the maximum circular speed of the subhalo {\em at the time it merged with the distinct halo}. \citet{conroy_etal06} demonstrated that a SHAM algorithm that assigns luminosities to subhalos based upon $\vacc$ can describe a broad range of galaxy clustering data from $z\approx 0$ to $z\approx 4$. More recently, \citet{reddick_etal12} showed that using $\vpeak,$ defined as the largest value of $\vl$ throughout the entire merger history of the subhalo, provides the most faithful joint reproduction of galaxy clustering data and the observed conditional stellar mass function. Our fiducial mock catalog is based on abundance matching with $\vacc,$ but we have also studied catalogs based on $\vpeak$ as well as $\vzero,$ the $z=0$ value of $\vl.$ 

Our SHAM procedure begins by using Eq.~(\ref{eq:lv}) to match the
distribution of luminosities assigned to dark matter halos and
subhalos to the double-Schechter function fit in
\citet{blanton_etal05}. We refer to these luminosities as $\linit.$
SHAM models with scatter between $\vl$ and $\mr$ more successfully
describe a variety of astronomical data \citep[see][and references
therein]{klypin_etal11,trujillo-gomez_etal11,watson_etal11} than
models with no scatter. Accordingly, we introduce scatter as follows. For the
$\ith$ halo in the catalog, we assign an independently chosen random
variable $\delta\mri$ drawn from a Gaussian distribution of width
$\sigmai.$ We use these random variables to assign new luminosities to
the galaxies in the catalog via $\mri\rightarrow\mri+\delta\mri.$ In
our fiducial catalog, we choose $\sigmai=0.5$ for all halos, which
introduces roughly $0.2$dex of scatter in the galaxy luminosities at the faint end of the luminosity function and $0.15$dex at the bright end,
which is very similar to the level of scatter used in \citet{trujillo-gomez_etal11}. We refer to these brightnesses as
$\lscatter.$

Our goal is to construct a mock catalog with a luminosity function
that exactly matches that of the Mr19 catalog, rather than the
\citet{blanton_etal05} luminosity function. To accomplish this, we
rank-order all the halos and subhalos in the simulation by their
luminosities $\lscatter.$ Because of the scatter we have introduced, this ordering of the halos is
non-monotonic in $\vl.$ 

Rank-ordering the observed Mr19 galaxies by their luminosity naturally provides a map from cumulative number density $\ngal(<\mr)$ to $\mr.$ We use this map to associate r-band magnitudes to halos
in Bolshoi. The $\ith$ halo in the list, ordered as described above,
is assigned a rank-ordered cumulative number density $\nrank\equiv i/\vbolshoi,$ where
$\vbolshoi=250(\hmpc)^3$ and $i$ is the rank-order of the $\ith$ halo. We use $\nrank$ to
assign luminosities to the halos by linear interpolation of the map
from $\ngal(<\mr)$ to $\mr.$ Halos with rank-ordered cumulative number
densities larger than $\ngal(<\mr=-19)$
are discarded.\footnote{There are only two Bolshoi halos with rank-ordered cumulative 
number densities less than the value of $\ngal$ of the brightest Mr19
galaxy. These halos are not reassigned a new luminosity, but keep
the $\lscatter$ value assigned to them by the initial (post-scatter)
abundance match to the \citet{blanton_etal05} luminosity function.} This procedure gives a
luminosity function of the mock galaxies that {\em exactly matches}
the Mr19 luminosity function, and which includes scatter in the
mapping between $\vl$ and $\mr.$ The reason for the initial abundance match to the \citet{blanton_etal05} analytical fit is simply that the exact luminosity function of galaxies dimmer than $\mr=19$ that are located in the spatial region occupied by the Mr19 galaxies is not known.

%The reason for the initial abundance match to the \citet{blanton_etal05} analytical  fit is as follows. For any brightness threshold $\lthresh,$ a $\vl-\mr$ map that properly includes scatter should result in a significant number of halos with $\linit$ dimmer than $\lthresh$ but $\lscatter$ brighter than $\lthresh.$ A SHAM prescription that first provides a monotonic match to the Mr19 luminosity function, discards those halos with $\linit$ dimmer than $\mr=-19,$ and then introduces scatter, would not have this feature.


%--------------------------
\section*{Appendix B}
\label{section:appendixb}
%---------------------------

Fiber collisions often occur when two or 
more galaxies are located within an angular separation of $55$
arcseconds from one another.  This angular separation is the 
minimum separation permitted by the width of the optical fibers used 
in the spectral measurements. When this occurs,
the fiber is positioned to measure the spectrum of a randomly chosen
galaxy from the two or more ``fiber-collided'' galaxies. In the data used 
to construct the DR3-based group catalog, the remaining galaxies of the fiber-collided set are 
assigned the redshift {\em and} brightness of the randomly-chosen galaxy.  
In the DR7-based group catalog that we utilize in this study, 
only the redshift of the randomly-chosen galaxy is assigned to the
remaining fiber-collided galaxies.  The absolute r-band magnitudes of 
the remaining fiber-collided galaxies are inferred from their apparent 
r-band magnitudes using the redshift of the randomly-chosen, 
spectroscopically-observed galaxy. This different treatment has important
 consequences for magnitude gap measurements, and in fact constitutes the primary reason for the difference between our conclusions and those drawn in \citet{paranjape_sheth11}. 
 
 The \citet{paranjape_sheth11} treatment of fiber collisions differs from ours in two important ways. First, we define $\monetwo$ to be the r-band magnitude difference between the two brightest non-fiber collided members of the group, whereas \citet{paranjape_sheth11} use fiber-collided galaxies in their $\monetwo$ definition. Second, recall from \S~\ref{section:data} that in the DR3 sample studied in \citet{paranjape_sheth11}, fiber-collided galaxies are assigned the r-band magnitude of their nearest neighbor on the sky. As a consequence of this, if one uses fiber-collided galaxies to define $\monetwo$ then all groups with a member that is fiber-collided with the brightest member are assigned magnitude gaps precisely equal to zero. This is a relatively common scenario since brightest group galaxies are typically found in the densest regions of the sky. As a result, with this treatment of fiber collisions combined with this definition of the magnitude gap, there is an artificial ``spike" at $\monetwo=0.$ The authors in \citet{paranjape_sheth11} attempted to account for this by discarding all groups with $\monetwo=0$ when performing their KS test. However, in addition to incorrectly enhancing the abundance of $\monetwo=0$ groups, this treatment of fiber collisions also results in an abundance of large- and moderate-gap groups that is systematically too low, since $100\%$ of such groups with a galaxy that is fiber-collided with the brightest member are incorrectly removed from large- and moderate-gap bins and assigned $\monetwo=0.$ As one can see by the sign of $\Psi(\monetwo|> N)$ in Fig.~\ref{fig:psigap}, the sense of this systematic error is such that it improves the agreement between $\Phi(\monetwo|> N)$ and its Monte Carlo relative to our treatment.

The $\monetwo=0$ spike and concomitant moderate- and lage-gap decrement does not occur in the DR7 treatment. {\em In our DR7 sample the gap abundance is the same regardless of whether not fiber-collided galaxies are used in the definition of $\monetwo,$ whereas in DR3 the gap abundance is very sensitive to this choice.} Moreover, the DR3 gap abundance when fiber-collided galaxies are excluded is in excellent agreement with DR7 gap abundance, and so we conclude that the treatment of fiber collisions in \citet{paranjape_sheth11} leads to incorrect measurements of $\Phi(\monetwo)$ that result in the erroneous conclusion that the observed gap abundance is consistent with the random draw hypothesis.


\end{document} 
t} 
