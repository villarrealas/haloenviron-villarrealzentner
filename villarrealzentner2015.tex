\documentclass[usenatbib,usegraphicx,letterpaper]{mn2e}
\usepackage[totalwidth=480pt,totalheight=680pt]{geometry}

\usepackage{amssymb}
\usepackage{epsfig}
\usepackage{amsmath}
\usepackage{color}
\usepackage[dvipsnames]{xcolor}
\usepackage{epsfig}  
\usepackage{graphicx}
\usepackage{subfig}
\usepackage{rotating}
\usepackage{array}
%%\usepackage{physics}

%simulation notation
\def\simA{L0125}
\def\simB{L0250}
\def\simC{L0500}

%Journals
\def\pasj{{PASJ}}
\def\nat{{ Nature }}
\def\aap{{ Astron. \& Astrophys. }}
\def\aj{{ Astron.~J. }}
\def\apj{{ Astrophys.~J. }}
\def\araa{{ Ann. Rev. Astron. Astrophys. }}
\def\apjl{{ Astrophys.~J.~Letters }}
\def\apjs{{ Astrophys.~J.~Suppl. }}
\def\apss{{ Astrophys.~Space~Sci. }}
\def\icarus{{ Icarus }}
\def\mnras{{ MNRAS }}
\def\pasp{{ Pub. Astron. Soc. Pacific }}
\def\planss{{ Plan. Space Sci. }}
\def\physrep{{ Phys. Rep.}}
\def\jcap{{J. Cosm. Astropart. Phys.}}

% less then similar, greater than similar
\def\lsim{\lower0.6ex\vbox{\hbox{$ \buildrel{\textstyle <}\over{\sim}\ $}}}
\def\gsim{\lower0.6ex\vbox{\hbox{$ \buildrel{\textstyle >}\over{\sim}\ $}}}

% begin equation
\newcommand{\beq}{\begin{equation}}
\newcommand{\eeq}{\end{equation}}
\newcommand{\beqa}{\begin{eqnarray}}
\newcommand{\eeqa}{\end{eqnarray}}

% basic cosmology
\newcommand{\Ho}{H_{0}}
\newcommand{\Om}{\Omega_{\mathrm{M}}}
\newcommand{\Ol}{\Omega_{\Lambda}}
\newcommand{\Ode}{\Omega_{\mathrm{DE}}}
\newcommand{\rhocrit}{\rho_{\mathrm{crit}}}
\newcommand{\Ok}{\Omega_{\mathrm{K}}}
\newcommand{\wzero}{w_{0}}
\newcommand{\wa}{w_{\mathrm{a}}}
\newcommand{\wpiv}{w_{\mathrm{piv}}}
\newcommand{\apiv}{a_{\mathrm{piv}}}
\newcommand{\ellmax}{\ell_{\mathrm{max}}}
\newcommand{\fsky}{f_{\mathrm{sky}}}

\newcommand{\fom}{\mathcal{F}}
\newcommand{\Rvir}{r_{\mathrm{vir}}}
\newcommand{\Rdel}{r_{\Delta}}

% units
\newcommand{\Msun}{\mathrm{M}_{\odot}~}
\newcommand{\hMsun}{\ h^{-1}\mathrm{M}_{\odot}~}
\newcommand{\hMpc}{\ h^{-1}\mathrm{Mpc}~}
\newcommand{\hkpc}{\ h^{-1}\mathrm{kpc}~}
\newcommand{\cpiv}{c_{\mathrm{piv}}}
\newcommand{\kmsmpc}{~\mathrm{km/s/Mpc}~}
\newcommand{\kms}{~\mathrm{km}~\mathrm{s}^{-1}}
\newcommand{\Mpc}{\mathrm{Mpc}}
\newcommand{\kpc}{\mathrm{kpc}}
\newcommand{\pc}{\mathrm{pc}}
\newcommand{\au}{\mathrm{AU}}
\newcommand{\gev}{\mathrm{GeV}}

% roman differential
%\newcommand{\dd}{\mathrm{d}}

% comments
\newcommand{\arz}[1]{{\color{BrickRed}\textbf{[ARZ: }\textbf{#1}]}}
\newcommand{\asv}[1]{{\color{TealBlue}\textbf{[ASV: }\textbf{#1}]}}


\bibliographystyle{mn2e}

%%%%%%%%%%%%%%%%%%%%%%%%%%%%%%%%%%%%%%%%%%%%%%%%

\begin{document}

\title[Halo Definition and Environmental Effects]{Halo Definition and Environmental Effects}
\author[Villareal et al.]
{Antonio Villarreal$^1$\thanks{E-mail: asv13@pitt.edu},
Andrew R. Zentner$^1$\thanks{E-mail: zentner@pitt.edu}, 
Christopher W. Purcell$^2$\thanks{E-mail: cwpurcell@mail.wvu.edu},\\
 Andrew P. Hearin$^3$\thanks{xxx},
 Frank C. van den Bosch$^4$\thanks{yyy}\\
$^{1}$Department of Physics and Astronomy \& \\
Pittsburgh Particle Physics, Astrophysics, and Cosmology Center (Pitt-PACC),\\ 
University of Pittsburgh, Pittsburgh, PA\\
$^{2}$Department of Physics and Astronomy, \\
West Virginia University, Morgantown, WV}

\date{In preparation}

%%\pagerange{\pageref{firstpage}--\pageref{lastpage}} \pubyear{2015}

%% \label{firstpage}

\maketitle

\begin{abstract}
%% Abstract goes here
Recent work has shown the importance of environment to the properties of dark matter halos. This brings conflict to standard implementations of the halo model and excursion set theory which assume that the properties of a population within the halo is determined by the mass of the halo alone. We seek to find a definition of the size of a halo that allows us to minimize the impact of assembly bias on halo model calculations. We analyze the dependence on environment of our properties using the method of marked correlation functions for several different halo definitions, utilizing the \citet{diemer15} simulations. We find that environmental dependencies are dramatically different as we vary the definition of the halo radius in terms of the overdensity $\Delta$. At large length scales, we find that the majority of assembly bias is removed through suitable redefinition of $\Delta$. We are able to determine that the majority of the reduction in assembly bias is related to the elimination of host halos that would cease to be hosts in catalogs at lower values of $\Delta$. Further, we analyze how different mass cuts affect this methodology. We note that unresolved halos leads to assembly bias being missed and that the most massive halos seem to exhibit minimal assembly bias. We further note that the choice of halo definition can induce assembly bias and consider how this may be interpreted in the context of previous results in the literature.
\end{abstract}

\begin{keywords}
dark matter -- galaxies: halos -- galaxies: formation -- large-scale structure of universe -- methods: numerical
\end{keywords}

%% notes on citation style:
%% \citep{stuff01,stuff02,stuff03} produces (Stuff 2001; Stuff 2002; Stuff 2003)
%% \citet{stuff04} produces "Stuff (2004)" in the main body
%% \\* defines a break in a section title it appears?
%% \begin{enumerate} into \item allows you to do the (i), (ii), (iii) thing 


\arz{We should add Andrew Hearin and Frank van den Bosch. Look into formatting the title correctly. Once I'm happy with the quality of the writing, we'll also need to send this to Benedikt Diemer and Andrey Kravtsov and offer them authorship.}

%-----------------------
\section{Introduction}
\label{section:introduction}
%-----------------------
 \arz{We will need to work on the introduction considerably as we get a better handle on the final results. 
 The first two paragraphs can probably be combined into a single shorter paragraph. I also like to end the first paragraph by telling the reader what it is that we aim to do in the paper. The rest of the introduction will need be be completely rewritten in a more professional manner. However, let's get the middle of the paper complete before we work a lot on the introduction and conclusion sections.
 Please pay close attention to your wording. As an example, it is MUCH preferable to say "galaxies and clusters form within 
 merging dark matter halos" than it is to say "the creation of observed galaxies and clusters is often seen as arising from the 
 hierarchical mergers of dark matter halos." The second option is just too long-winded without adding any information.}

In the current concordance cosmology, the creation of observed galaxies and clusters is often seen as arising from the hierarchical mergers of dark matter halos.\citep{white78}. \arz{Add Blumenthal et al. citation here.} 
Being able to model the properties of dark matter halos and the galaxies within gives us a potential probe for the physical processes that go into galaxy formation. The excursion-set formalism of galaxy clustering \citep{bond91,lacey93,somerville99, zentner06} and the standard halo model of galaxy clustering \citep{seljak00, peacock00, scoccimarro01, berlind02, bullock02, cooray02} both can help us in this task, but rely on underlying asumptions. The first is that the statistics of the objects within a dark matter halo is a function of the mass alone. The second is that the clustering of dark matter halos is a function of mass as well. In this paper we propose a simple redefinition of halo size that will help correct for inaccuracies in these two assumptions.

It has previously been demonstrated that the clustering of halos is dependent on not only the halo mass, but also on the formation time of the halo \citep{sheth04, gao05, wechsler06, croton07, zentner07}. Furthermore, it has been shown that the clustering of a given halo is dependent on the halo concentration \citep{wechsler06}. \arz{Using this twice! argh! Such a sentence is very awkward and unclear. Generally, avoiding using "this" or "that" frequently unless it is absolutely obvious what "this" and "that" are.} This necessitates corrections be made to the standard implementations to account for this. More complicated methodologies have been made to extend to using merger histories directly from simulation \citep{dvorkin11}, as well as to account for the dependence due to concentration \citep{gil11}. The relationship that clustering has to the properties of the halo is commonly referred to assembly bias or environmental effects.

Our method of halo redefinition is motivated by the size of a halo not necessarily being a well-motivated property. What is often referred to as the ``virial radius'' will not contain all gravitational bound dark matter particles in the halo \citep{kazan06}. Rather, it is a matter of convention that does not have a common definition across the literature. For example, some studies have chosen to use halo radius defined with respect to the critical mass density of the universe, while many simulation papers choose to use the mean background mass density of the simulation. Furthermore, the overdensity often used can typically vary from 178 (from basic spherical collapse) to 200 (a common simulation definition) to 337 (``virial'' in $\Lambda\mathrm{CDM}$). This can lead to a change of up to tens of percent in the measured halo radius from one measurement to the other. We choose to define a halo radius in a way that encompasses the nearby environmental effects. These may be due to large scale structure or driven by baryonic physics. Using a simulated box, we can then test how the redefinition of the halo size affects the relationship between the clustering and the properties of the halo. In the case that the properties of the halo become independent of the clustering, it is possible to utilize standard implementations of the halo model without necessitating more complicated theory.
 
In \S~\ref{section:data} of this paper, we discuss the simulated cosmological boxes that we utilize for our statistics and the halo finder utilized. In \S~\ref{section:haloprops}, we consider the properties of interest within our halo simulation and their standard definitions. In \S~\ref{section:methodology}, we discuss the statistics that we have used in order to test for environmental effects and the removal of known mass scaling from these statistics. In \S~\ref{section:results}, we present our results on our tests and consider how each definition affects assembly bias. In \S~\ref{section:conclusions}, we discuss the significance of reducing environmental effects through a redefinition of halo environments and discuss possible applications of our methodology. We also consider the nature of assembly bias as a function of halo definition.

%------------------------------------
\section[]{Simulations and Halo Identification}
\label{section:data}
%------------------------------------

In order to study the effects of halo redefinition, we use three cosmological $N$-body simulations of structure formation. The \citet{diemer15} simulations each utilize a Planck best-fit cosmology with $\Om = 0.32$, $\Ol = 0.68$, and $h_{\mathrm{o}} = 0.67$. We use three simulation boxes with comoving sizes of 125, 250, and 500 $\hMpc$ respectively. The particle masses are $1.6 \times 10^8$, $1.3 \times 10^9$, and $1.0 \times 10^{10} \hMsun$ respectively, implying a total of $1024^3$ particles in each simulation. Furthermore, the three simulations have different force softening scales of $2.4$, $5.8$, and $14 \hkpc$. We refer to each simulation as \simA, \simB, or \simC  \ for the remainder of the paper. This set of simulations allows us to probe the resolution effects inherent in halo finding (due to the varying resolutions of the simulations) and to probe the mass dependence of halo clustering over a wider range of halo masses than would be possible with only one simulation from the set. In particular, \simA~, with its higher resolution, contains the least massive resolved halos, while \simC~has the most robust statistics for the most massive halos as a result of the larger simulation volume.

To identify halos, we use the ROCKSTAR halo finder, which works on the phase space algorithm described in \citet*{behroozi13}. In short, ROCKSTAR determines the initial groupings using a Friends-of-Friends algorithm in phase space before applying the spherical overdensity halo definition in order to determine halo properties of interest. Unbound particles are removed prior to the calculation of halo mass and other halo properties. 

\arz{The following sentences here seem out of place.}
In addition, we take interest of the shape of the halo, which is determined through the sorted eigenvalues of the inertia tensor, with principal ellipsoid axes defined such that $a >b > c$. Some of these parameters are determined after fitting the particles to the Navarro-Frenk-White profile \citet*{nfw97}.

\arz{Try this paragraph again. It is rambling. Also it has structural flaws such as referring to $\Delta$ before defining $\Delta$.} 
While the definition of halo size is fairly straightforward, the individual choices that make up the calculation are not particularly well-motivated. While the virial radius of a halo may be one natural choice, simulations have shown that it often misses gravitationally bound particles \citep*{kazan06}. Halo size has become a matter of definition, with $\Delta=200$ being the most common choice within the literature. However, it is not uncommon to see calculations carried out utilizing $\Delta = 178$, as derived through spherical collapse models, or $\Delta = 337$, as calculated in the case of $\Lambda\mathrm{CDM}$. Given the lack of a well-motivated standard, we choose to redefine the halo in terms of the overdensity parameter, $\Delta$, by defining the halo radius, $\Rdel$, as follows:
\beq
	\bar{\rho}(\Rdel) = \Delta\, \rho_{\mathrm{m}}
\eeq
Here, $\rho_{\mathrm{m}}$ is the mean background mass density of the simulation box, while $\bar{\rho}$ is the mean density within $\Rdel$. It should be noted that the use of the critical density or the mean background mass density is interchanged within the literature. The difference between these two values can be as high as a factor of a few, which will then carry forward into the calculation. We allow the overdensity parameter $\Delta$ to vary over the range of the fiducial definition of $\Delta = 200$ down to values as low as $\Delta = 10$. As discussed above, this change is motivated due to the fact that the definition of halo size is primarily a matter of convention.

%-------------------------------------
\section{Halo Properties}
\label{section:haloprops}
%-------------------------------------

We explore the clustering of halos as a function of a number of halo properties that are commonly explored in the 
literature and can be easily measured for a halo from an individual simulation snapshot. The first of these is the 
halo concentration parameter, 
\beq
c_{\mathrm{NFW}} = \frac{\Rdel}{r_\mathrm{s}},
\eeq
where $\Rdel$ is the radius of the halo given an overdensity parameter $\Delta$ used to define the halo and 
$r_{\mathrm{s}}$ is the halo scale radius. \arz{The scale radius has not yet been defined. How will 
the reader know what it is? How is it computed? You should probable introduce the NFW profile here 
rather than above (see my earlier comment on that).}
We also use an additional measure of the concentration of the halo density profile, 
\beq
c_{\mathrm{V}} = \frac{V_{\mathrm{max}}}{V_{\Delta}}, 
\eeq
where $V_{\mathrm{max}}$ is the maximum circular velocity achieved within the halo and $V_{\Delta}$ is the circular 
velocity at the halo radius, $\Rdel$. The quantity $c_{\mathrm{V}}$ can be measured from simulations without 
any need for fitting halo density profiles to determine scale radii and is therefore robust to choices of halo profiles 
and fitting methods. 

Halo concentrations are useful to explore for these purposes for a number of reasons. First of all, environment dependence 
of concentrations is of direct interest in modeling galaxy clustering and gravitational lensing statistics. Secondly, 
concentrations can be measured 
from individual simulation snapshots relatively easily yet halo concentrations are known to be strongly correlated with the formation 
histories of dark matter halos with earlier forming halos having higher concentrations at fixed halo mass 
\arz{Cite papers here like Wechsler et al. 2002}. As such, exploring the concentration dependence of halo clustering may 
yield insight into the age dependence of halo clustering without the need for constructing merger trees. This 
is particularly important in the present study in which the halo finding is performed repeatedly and constructing a 
merger tree for each run of the halo finder with different $\Delta$ can be prohibitive. We will explore measures of 
halo age directly in a forthcoming follow-up study dedicated to halo formation histories.


Figure~\ref{fig:cnfwrelation} shows the mean $c_{\mathrm{NFW}}$-$M_{\Delta}$ relation for halos defined with $\Delta=200$ 
in \simA, \simB, and \simC. For each simulation, we consider halos only above a minimum mass, as shown in Fig.~\ref{fig:cnfwrelation}, to ensure that concentration measurements are not compromised by resolution. Likewise, Figure~\ref{fig:cvrelation} shows the analogous 
relation for the alternative definition of halo concentration, $c_{\mathrm{V}}$. 

%------------------------------------------ Figure for Cnfw(M)
\begin{figure}
\centering
\includegraphics[width=.5\textwidth]{masscut_cnfw_d200.pdf}
\caption{An example of the relationship between the NFW concentration and halo mass for each of our simulations for $\Delta =200$. The chosen lower limit on halo mass for our sample is marked as a blue, red, or cyan line for \simA, \simB, and \simC \ respectively. At lower mass, halos are likely ill defined due to resolution limits.}
\label{fig:cnfwrelation}
\end{figure}
%--------------------------------------------------------------------------------


%------------------------------------------ Figure for C_V(M)
\begin{figure}
\centering
\includegraphics[width=.5\textwidth]{masscut_cV_d200.pdf}
\caption{An example of the relationship between the velocity ratio concentration and halo mass for each of our simulations for $\Delta =200$. The chosen lower limit on halo mass for our sample is marked as a blue, red, or cyan line for \simA, \simB, and \simC \ respectively. At lower mass, halos are likely ill defined due to resolution limits.}
\label{fig:cvrelation}
\end{figure}
%--------------------------------------------------------------------------------

Figure~\ref{fig:concentrations} shows the relationship between $c_{\mathrm{NFW}}$ and $c_{\mathrm{V}}$ for quantifying halo 
concentration on a halo-by-halo basis. Generally the two proxies for concentration are related to each other in a simple manner 
with relatively small scatter indicating that these two quantities largely contain the same information about each halo.

%-------------------------------------------- Figure comparing Cnfw and Cv on a halo-by-halo basis
\begin{figure}
\centering
\includegraphics[width=.5\textwidth]{L0250_compare_cnfwvcV_z00.pdf}
\caption{
The relationship between the two different marks of concentration, using halos in \simB. 
\arz{This is not sufficiently specific as a caption. For example, you don't say what the color code is 
or what the color scale is.}
}
\label{fig:concentrations}
\end{figure}
%--------------------------------------------------------------------------------------------------------------------------------

In addition to halo concentrations, we explore halo clustering as a function of a variety of other 
halo properties. We explore halo clustering as a function of halo shape quantified by the ratio of 
the minor and major axes length, 
\beq
s = \frac{c}{a},
\eeq
where $a$ is the major axis length and $c$ is the minor axis length. The mean relations of halo shapes 
as a function of halo mass for $\Delta=200$ is shown in Figure~\ref{fig:srelation} along with the mass 
thresholds selected to ensure that our results are not compromised by resolution.


%------------------------------------------ Figure for s(M)
\begin{figure}
\centering
\includegraphics[width=.5\textwidth]{masscut_shape_d200.pdf}
\caption{
An example of the relationship between halo shape and 
halo mass for each of our simulations for $\Delta =200$. The chosen 
lower limit on halo mass for our sample is marked as a blue, red, or 
cyan line for \simA, \simB, and \simC \ respectively. At lower mass, 
halos are likely ill defined due to resolution limits.
}
\label{fig:srelation}
\end{figure}
%--------------------------------------------------------------------------------

We also explore halo clustering as a function of halo spin quantified 
by the spin parameter $\lambda$ as introduced by \citep{peebles69},
\beq
\lambda = \frac{J \sqrt{\lvert E\rvert}}{G M_{\Delta}^{2.5}}
\eeq
where $J$ is the halo angular momentum, $E$ is the total energy of the 
halo, and $M_{\Delta}$ is the mass at the halo radius, $r_{\Delta}$. 
The mean relations of halo spins as a function of halo mas for $\Delta=200$ 
is shown in Figure~\ref{fig:spinrelation} along with the mass thresholds 
selected to ensure that our results are not compromised by resolution.

%------------------------------------------ Figure for lambda(M)
\begin{figure}
\centering
\includegraphics[width=.5\textwidth]{masscut_spin_d200.pdf}
\caption{
An example of the relationship between halo spin $\lambda$ 
and halo mass for each of our simulations for $\Delta =200$. 
The chosen lower limit on halo mass for our sample is marked 
as a blue, red, or cyan line for \simA, \simB, and \simC \ respectively. 
At lower mass, halos are likely ill defined due to resolution limits.
}
\label{fig:spinrelation}
\end{figure}
%--------------------------------------------------------------------------------

In practice, the mean relations between the various halo properties and the mass thresholds must be determined 
for each combination of simulation, halo property (e.g., $c_{\mathrm{NFW}}$ or $s$), and halo definition (e.g., value 
of $\Delta$). We use thresholds determined by the particular case under consideration so as to ensure that resolution 
limitations do not drive any of our primary results. A higher mass threshold means that we do not have any 
issues due to halo resolution, but of course, higher mass thresholds reduce statistical power. As an example, we 
summarize the mass thresholds we have used for a subset of $\Delta$ values in Table~\ref{table:thresholds}. 
At most values of $\Delta$, the minimum mass thresholds are driven by the requirement that the shape parameter, $s$, 
not suffer significantly from finite resolution effects. \arz{Please check the veracity of this last sentence. Some such 
sentence summarizing the factors that limit our mass cuts should be in the paper.}

\arz{In the table, don't ``low mass," ``mid mass," and ``high mass" correspond to the different simulations? For example, 
low mass corresponds to \simA~and so on? If not, then I don't understand. If so, then 
you should label each cut by the relevant simulation name as well.}

%%%%%%%%%%%%%%%%%%%%%%%%%%%%%%%%%%%%%%%%%%%%%%%%%%%%%%%%%%%%%%%%%%%%%%%%%%%%
\begin{table*}
\caption{
Minimum mass thresholds for each of our analyses depending upon the 
value of the overdensity, $\Delta$, used to define the halos. 
In the columns below the values of $\Delta$ we show the minimum 
host halo masses considered in units of $h^{-1}\mathrm{M}_{\odot}$.
}
\vspace*{8pt}
\begin{tabular}{ c c c c c }
\hline
\hline
Cutoff Name \ \  & $\Delta=200$ & $\Delta=100$ & $\Delta=75$ & $\Delta=50$ \\
\hline
\\{low mass} & $7 \times 10^{10}$ & $8 \times 10^{10}$ & $9 \times 10^{10}$ & $1 \times 10^{11}$  \\
{mid mass} & $7 \times 10^{11}$ & $8 \times 10^{11}$ & $9 \times 10^{11}$ & $1.5 \times 10^{12}$ \\
{high mass} & $4 \times 10^{12}$ & $5 \times 10^{12}$ & $6 \times 10^{12}$ & $7 \times 10^{12}$ \\
\hline
\hline
\end{tabular}
\label{table:thresholds}
\end{table*}
%%%%%%%%%%%%%%%%%%%%%%%%%%%%%%%%%%%%%%%%%%%%%%%%%%%%%%%%%%%%%%%%%%%%%%

\arz{I would like to see another figure here. I would like to see $M_{200}$ on the x-axis and $M_{\Delta}$ on the y-axis. 
What should be plotted is the mean $M_{\delta}$ as a function of $M_{200}$ for all halos common to both catalogs. This 
will be a nice informative plot for the reader and will help the reader to digest the table.}

%-----------------------
\section[]{Halo Clustering as a function of Auxiliary Halo Properties}
\label{section:methodology}
%-----------------------

\arz{I actually don't understand what you are saying in this paragraph. I hope there is nothing special about \simB~? 
Aren't the mass limits determined as you stated in the previous section for each simulation and for each $\Delta$? If so, 
why does this all need to be repeated here? Perhaps you can just get rid of this paragraph if you are giving this information 
in the previous section. That would be my preference for logical flow.}
There are several well understood effects in our simulation that must be accounted for before we can draw any conclusions. The first is the matter of simulation resolution. In each of our generated halo catalogs, we can anticipate finding artificial halos. While existing under our constraint of having only a set overdensity, they will have properties that conflict with well known trends that have been shown in prior works, such as concentration decreasing as a function of halo mass \citep{wechsler06}. As shown in Figure~\ref{fig:cnfwrelation}, we can roughly identify the region in which these artificial halos become significant by determining where the monotonic relationship between concentration and mass breaks at low mass. In order to avoid our halo finding statistics being thrown off by these artificial halos, we set a lower mass threshold, limiting the sample to only those halos that are physically significant. We have chosen our threshold to be set by \simB \ in order to minimize the shift to the remaining catalogs.

%----------------------------------------------------------
\subsection{Auxiliary Halo Properties}

\arz{I thought that the Duffy et al. paper only looked at concentration? Is this correct? Either way, there is an 
Allgood et al. paper that comprehensively examines this for various halo properties and should be cited here.}
Another well known effect is the scaling of our properties as a function of halo mass, as well demonstrated within the literature \citep{duffy08}. We are interested in studying the clustering behavior of halos as a function of properties other than mass. As mass is the dominant halo property correlated with halo clustering strength and environment, we refer to the additional properties that we study as 
``auxiliary halo properties" (those properties other than mass, such as concentration $c_{\rm NFW}$ or shape $s$). 

Most contemporary cosmological $N$-body simulations, including the set that we study here, 
do not have a sufficiently large number of halos to make isolating halos of fixed mass, and then further 
splitting these halos by an auxiliary property, a powerful method with which to study the dependence of 
clustering on auxiliary properties. Consequently, we study this phenomenon using halo samples 
selected according to minimum mass thresholds. To do so, we remove the mass dependence of 
the auxiliary properties as follows. 

We take all host halos of interest and sort them by their halo mass, $M_{\Delta}$. 
Each set of halo properties is then placed in bins of equal population in order to assure 
that enough data points exist for robust statistics. We then normalize each mark in a 
given bin to the mean value. The end result is halo properties that do not 
change as a function of mass, allowing us to more accurately analyze clustering behavior within our simulation. 
\arz{First, you need some equations here. It is unclear to the reader what the word ``normalize" means 
in this context. YOU MUST provide enough information so that someone with access to these simulations 
could reproduce your calculations. Also, you don't give important details. How many are in the ``equal populations" 
for example. You may also want to add an additional paragraph speculating briefly on other ways to 
remove the gross mass dependence. You must also say explicitly that we use the mass-dependence-removed 
concentrations, shapes, and spins in all following calculations. This might not be the exact wording you 
want to use, but you can decide what you want to call these things.}

\arz{Flesh this out a bit. See the discussion in Wechsler et al. 06 as a model. The specific 
values of the thresholds need to be given}
To mitigate the correlation between mass satellite number, 
we follow the prescription of \citet{wechsler06}. We select only satellite halos 
with $v_{\mathrm{max,sub}} \ge number$. This ensures that the satellite counts 
are minimally affected by finite resolution. We select subhalos using 
maximum circular velocity, $v_{\mathrm{max,host}}$ because it is more robustly 
determined for subhalos and more easily enable comparisons with other working 
undertaken using distinct halo identification algorithms. 
We then choose only host halos that are sufficiently 
large to contain, on average, $number$ such satellite halos. This corresponds to a cut on host halo 
maximum circular velocity of $v_{\mathrm{max}} \ge number$. Finally, we enumerate for each 
host, the number of subhalos within the radius $R_{\Delta}$ of the host above a threshold 
in the ratio $v_{\mathrm{max,sub}}/v_{\mathrm{max,host}} \ge number$. The subhalo velocity 
function is a very nearly self-similar function \arz{Cite Zentner, Berlind et al. 2005 and references therein 
on this point.}, so that scaling subhalo $v_{\mathrm{max,sub}}$ by 
host $v_{\mathrm{max,host}}$ in this way eliminates the 
gross mass dependence of satellite number.
\arz{In the earlier draft, the preceding paragraph had a logical flow that made it 
very difficult to follow the steps in the calculation. If you think about it for a minute, the 
important pieces are: (1) we quantify subhalo size with $v_{\mathrm{max,sub}}$ because 
it is more robustly measured than subhalo mass; (2) there is a minimum $v_{\mathrm{max,sub}}$ 
that we can aspire to reach because of resolution; (3) we quantify subhalo number by scaling 
subhalo size relative to host size, $v_{\mathrm{max,sub}}/v_{\mathrm{max,host}}$, because this 
scales out the gross dependence of subhalo number count on host halo size; (4) we must choose 
host halos that are sufficiently large that the contain, on average, a few subhalos above our 
minimum subhalo $v_{\mathrm{max,sub}}$. These considerations completely specify the 
cuts on $v_{\mathrm{max,sub}}$ and $v_{\mathrm{max,host}}$. Please make sure these 
elements of the logic are clear to the reader.}

%---------------------------------------------------------------
\subsection{Clustering Statistics}

We assess the influence of assembly bias on two-point statistics of host halos. In order to do so, we 
study both the standard two-point correlation functions of halos selected by properties other than mass 
(e.g., the auxiliary properties concentration, shape, and spin) as well as halo mark correlation functions (MCFs). 
MCFs quantify the manner in which a halo property (the ``mark") correlates among halo pairs as a function of 
the distance between the pairs. Absent halo assembly bias, halo marks are uncorrelated among pairs. 
MCFs have been used in several previous papers to quantify environmental dependence of halo 
properties other than mass \citet{wechsler06} or \citet{harker06} \arz{I think there are also a couple of 
Sheth papers that should be cited here, maybe more}. 
For a specific halo property, or mark $m$, we use the MCF normalization of \citet{wechsler06}, namely 
\beq
\mathcal{M}_m(r) \equiv ( \langle m_1 m_2 \rangle_p (r) - \langle m \rangle^2 / \mathcal{V}(m),
\eeq
where $m_{\mathrm{i}}$ is the value of the mark for halo $\mathrm{i}$, $\langle m \rangle$ is the mean of the mark, 
and $\mathcal{V}(m)$ the variance of the mark. The notation is intended to indicate that the average is taken over all 
pairs of halos separated by a distance $r$. In the absence of any correlation between a halo property and the halo 
having a neighbor a distance $r$ away, $\mathcal{M}_m(r) = 0$. Deviations of the MCF from zero indicate 
such correlations exist and the size of the $\mathcal{M}_m(r)$ gives the excess of the mark among pairs 
compared to the one-point mean of the mark $\langle m\rangle$ in units of the one-point variance.


It is necessary to assess statistical fluctuations in the statistics that we measure in these simulations in order to 
determine the significance of the signals. For two-point correlation functions, we determine the covariance of 
the measurement through jackknife resampling of the eight octants of the simulation cube. We assess the 
statistical significance of the MCFs by randomly re-assigning each of our marks to the halos in the sample. 
We then compute the MCF of these randomized marks. As the mark re-assignment is random, 
the MCF computed on these re-assigned marks {\em cannot} exhibit any 
environment dependence other than that induced by statistical fluctuations. 
We perform this reassignment 400 times and approximate a $2\sigma$ error region by 
the span of the MCF between our 10$^\mathrm{th}$ lowest and 10$^\mathrm{th}$ highest (that is the 390$^\mathrm{th}$ 
if the MCFs were sorted in ascending order) values of the MCF. In the event that there are no environmental 
correlations with halo auxiliary properties, the MCFs measured in the simulations would fall within this error 
band $95\%$ of the time. 


\arz{The sentences immediately below this comment can be removed. Also, not a wise idea to 
lock yourself into 20\% in the text. There is absolutely nothing special about 20\%. In fact, it 
would be interesting to see if things change as this percentile changes, perhaps try 50\% and 
10\%. We should at least know what those figures look like before proceeding to publication, 
so you should make them even though we likely won't include them in the main text of 
the paper. It could be very informative and will almost certainly be useful when giving talks.}
The first is to look at the two-point correlation functions of the host halo catalogs. We take the difference between the samples of the top 20\% and bottom 20\% in normalized concentration, and then normalize these results by the overall correlation function.. Assuming that each halo was statistically similar, regardless of host halo clustering, we would expect that this statistic would approach zero at all scales.



%----------------------------
\section[]{Results}
\label{section:results}
%----------------------------

\arz{We don't need to reiterate all of this stuff about the mass cuts. Just make sure the discussion in the simulations section is complete and then we can move forward. Again, I suggest changing the names of your cuts. It would be great if they can signify something either about the ACTUAL MASS involved in the cut or the simulation or possibly BOTH! Perhaps you can say the ``L0125" cut and so on? 
If you get everything specified in the simulation section, then this paragraph is completely unnecessary.}
The ``mid mass" set of cuts has been chosen for our preliminary analysis choice, as it gives us a healthy balance between mass resolution and robust statistics. Figure~\ref{fig:cnfwrelation} shows an example of this cutoff in the $\Delta = 200$ case as a red line. 
\arz{It seems to me that we DO include \simC~!! Even if we don't we should. Let's treat each simulation as simply sampling a different 
range of halo masses. That is the most useful way to proceed.}
We do not include \simC \ in this initial analysis, as that simulation holds several bins of halos deviating significantly from a monotonic mass-concentration relation. This may be a result of including halos that are unphysical, but still meets the halo-finding algorithm's requirements.  The remaining two cuts have been chosen to explore any resolution related effects. The ``low mass'' cuts explore the lowest possible mass cutoff that excludes ill resolved halos in the \citet{diemer15} simulations, while the ``high mass'' set of cuts represent the mass cutoff chosen to avoid contamination due to resolution effects in our largest simulation box.

%-----------------------------------------------
\subsection{Correlation Functions}
\label{sub:cfresults}

\arz{Fix up and fill in numbers as needed, but follow this general format.}
We begin by studying the correlation functions of halos in our mass threshold samples, sub-selected by auxiliary properties. 
Figure~\ref{fig:cc_cfcompare} exhibits the difference between the clustering strength of halos in the $20^{\mathrm{th}}$ percentile highest concentrations and the halos with the halos that have the $20^{\mathrm{th}}$ percentile lowest concentrations as a function of the overdensity parameter $\Delta$, used to define the halos. First, it is clear that clustering strength generally depends upon the auxiliary properties of halos at all halo mass thresholds. Second, it is clear that the strength and sign of assembly bias is 
strongly mass dependent, a result that agrees with the significant previous literature on halo assembly bias \arz{Cite all the usual 
papers here, Gao, Wechsler, Harker, Zentner07, Dalal07, probably several more.} At relatively low mass (the L0125 panel, 
$M_{200} > number$), high-concentration halos are considerably more strongly concentrated 
than low-concentration halos using the more 
conventional $\Delta = 200$ definition for halos. At somewhat higher halo masses (the L0250 panel, $M_{200} > number$), this 
difference is markedly reduced. Finally, for the highest mass halos that we have the capability of studying (the L0500 panel, 
$M_{200} > number$), the effect is of opposite sign; low-concentration halos are more strongly correlated than high-concentration halos.

\arz{The language in this paragraph is way too loose and not sufficiently clear or professional. What does it mean to be a ``reasonable" mass range, for example? I'll attemp to clean it up. Try to follow this model in your writing. Compare my version:}
Focus first on the middle panel of Fig.~\ref{fig:cc_cfcompare}. In this panel, corresponding to the 
\simB~mass threshold, the difference in large-scale clustering between high- and low-concentration halos 
is dramatically reduced for a halo definition with $\Delta=75$. Further decreasing $\Delta$ leads to concentration-dependent 
clustering of opposite sign. Comparing the differing clustering 
strengths across the three panels of Fig.~\ref{fig:cc_cfcompare}, it is clear that this is not a universal conclusion. 
For low-mass halos, very low values of $\Delta$ (and correspondingly large definitions of halo radii, as $R_{\Delta} \propto \Delta^{-1/3}$) 
are necessary in order to mitigate the concentration dependence of halo clustering. Conversely, for higher mass halos (the L0500 panel), 
values of $\Delta \approx 200$ yield little concentration-dependent clustering. In this case, decreasing $\Delta$ (increasing $R_{\Delta}$) results in significantly {\em increased} concentration-dependent halo clustering. The reasons for these changes is of interest and we return to interpreting these results below.
\arz{To your version:}
We see that when we are interested in studying primarily a sample of low mass halos, such as in in \simA, it is difficult to use this method to account for assembly bias for any reasonable matter. However, when looking at a more reasonable mass range, there is a removal of assembly bias on large scales between $\Delta = 75$ and $\Delta=50$ that can be noted in the data. It should be noted that this is effectively a ``sweet spot'' effect, where decreasing the value of $\Delta$ (or correspondingly, the size of an average halo) will start to give increasing amounts of power in the correlation function to the least concentrated halos - essentially introducing a new assembly bias. The physical nature of this is of interest to us, and we attempt to present an intuitive sense of the effect further on. Furthermore, we note that when interest is focused on the most massive halos, assembly bias appears to be minimal in terms of the clustering for the largest scales with the standard definition, perhaps giving a more statistical motivation to the use of $\Delta = 200$.

Notice that in all panels of Fig.~\ref{fig:cc_cfcompare}, the effect of concentration-dependent clustering is scale dependent. 
Moreover, the effect is scale-dependent for all values of $\Delta$. In these cases, simply defining halos with a different value 
of $\Delta$ does not suffice to eliminate concentration-dependent clustering on all scales. In this discussion and throughout, 
we focus primarily on the large scale clustering, which we take to mean clustering on scales significantly larger than the radii, 
$R_{\Delta}$ of the halos in our samples. In each panel in Fig.~\ref{fig:cc_cfcompare}, we designate the radii of the largest halos 
in each sample with vertical dashed lines.

\arz{Given how the correlation functions look, shouldn't we include results for $\Delta<50$, for the L0125 results? The 
effect is continuous so it looks like it would be removed with a smaller value of $\Delta$, say $\Delta \approx 20$. I think you have 
run this already, so no reason to leave it out. I say this is a must for moving forward 
with publication.} 

\arz{On another note, did you ever compute these things for finer spacing of $\Delta$? For example, it looks like $\Delta \approx 70$ 
is going to work better than $\Delta=75$. It's OK if you haven't, but let me know! I thought we had discussed this. }

\arz{For the high-mass, L0500, cut, did you try any $\Delta>200$, maybe $\Delta=210$ or so? This is not really necessary, 
I'm just curious how this looks.}


\begin{figure}
	\centering
	\includegraphics[width=.4\textwidth]{all_cfhilow_z00_cutcomp.pdf}
	\caption{
\arz{All lines here could stand to be slightly thicker.}
The difference of the correlation function for only the top 20\% most concentrated halos and the bottom 20\% in concentration, normalized by the overall correlation function of the entire sample. From top to bottom we show the results for the lowest mass cut in \simA, \simB, and \simC \ that excludes ill resolved halos. \arz{This method of specifying the mass cuts is poor style. You already gave each mass cut a name in the table. Why now use a different way of referring to the mass cuts? This confuses the reader, and even I find it confusing. I'm not sure what the plot shows! Just use the names of the cuts!} The dashed lines along the bottom denote the largest halo radius for a given value of the overdensity parameter. \arz{This figure needs to be remade. Labels need to be cleaned up. Each panel needs to be labeled with simulation name and mass threshold (as an aside, I prefer the word ``threshold" to ``cut"). I also think that you could stand to make the figure a little bit bigger. Not a big deal to me. Lastly, this figure doesn't give a good sense of the statistical 
error. How about putting error bars on one of these lines? Perhaps on the $\Delta=75$ line in the middle panel. That is the most important one to have error bars on.}
	}
	\label{fig:cc_cfcompare}
\end{figure}

\arz{I assume that if you made the same plot for $c_V$ we would reach the same basic conclusions as for $c_{NFW}$? If this is so, then you need to add a specific statement in this regard to the end of the discussion above. In this case, no new figure would be needed. If not, then you need to show the figure.}

\arz{In this subsection, I advocate at least one and perhaps two new figures. First, can you make an analogous plot to Fig.~\ref{fig:cc_cfcompare} that shows some other property, such as shape or spin? This would be a nice addition for completeness. 
Second, what would happen if you used 50\% or 10\% cuts on concentration? Would we reach nearly the same conclusions? 
If so, then you do not need to show the plots explicitly. However, in this case, you DO need to state explicitly that this is the case. 
In other words, you need to state the the broad conclusions we reach are roughly independent of the specific percentiles that we 
use to select halos so that showing results for a variety of different percentiles doesn't add very much to the discussion. It is also 
possible that selecting on a different percentile yields an entirely different conclusion. I don't suspect that this will be the case, but 
it is possible. If that happens, then you need to show an example of such a figure for, say, concentration.}

%-----------------------------------------------------------------
\subsection{Mark Correlation Functions}


%% note: we need to remake this plot!

\begin{figure}
	\centering
	\includegraphics[width=.4\textwidth]{all_mcf_cnfw_z00_cutcomp.pdf}
	\caption{
\arz{Many of the same comments from the previous figure are also relevant here, label mass cuts/simulations in each panel etc.}
The marked correlation function for the concentration defined according to the NFW profile. From top to bottom we show the results for the lowest mass cut in \simA, \simB, and \simC \ that excludes ill resolved halos. The shaded bands represent 2-sigma confidence regions generated by randomization of the marks. The dashed lines along the bottom denote the largest halo radius for a given value of the overdensity parameter.}
	\label{fig:cc_mcf_cnfw}
\end{figure}


We now move toward a discussion of halo assembly bias as diagnosed by MCFs. 
MCFs have the advantage that it is not necessary 
to specify particular auxiliary property subsamples, such as the percentiles above, 
in order to assess assembly bias\footnote{Of course, this comes at the cost of averaging 
over all values of the auxiliary properties. Using MCFs it is not evident if the assembly bias effect 
is a smooth function of the auxiliary properties, dominated by the tails of the auxiliary property, or 
has some more complex dependence upon the auxiliary property. Our results in \S~\ref{sub:cfresults} suggest 
that assembly bias is a fairly smooth function of the auxiliary properties.} 
The NFW concentration, $c_{\mathrm{NFW}}$, MCF is shown in Figure~\ref{fig:cc_mcf_cnfw}. 
The shaded bands in the figure delineate the statistical fluctuations in MCFs induced by 
finite sampling as discussed in the previous section. Qualitatively, 
Fig.~\ref{fig:cc_mcf_cnfw} exhibits the same features that are evident in 
Fig.~\ref{fig:cc_cfcompare}: more concentrated halos are significantly more clustered in 
the low-mass L0125 halo sample; concentration-dependent halo clustering weakens and 
reverses sense as halo mass increases; for the L0250 sample with $\Delta=70$, the large-scale 
concentration dependence of halo clustering has been reduced so as to be consistent with 
zero within the statistical limitations of the simulation. \arz{Again, it seems like we need a 
$\Delta <50$ line for L0125. If possible, it would be great if we could put a $\Delta>200$ line 
on this plot for the L0500 simulation.}


Figure~\ref{fig:cc_mcf_cV} is a similar plot for the velocity-defined concentration, $c_{\mathrm{V}}$ MCF. 
This figure exhibits qualitatively and quantitatively similar features to Fig.~\ref{fig:cc_mcf_cnfw}, a 
fact that is not surprising given that we already know that $c_{\mathrm{NFW}}$ and $c_{\mathrm{V}}$ 
quantify largely redundant information about their halos. 

%---------------------------------------------------
\begin{figure}
	\centering
	\includegraphics[width=.4\textwidth]{all_mcf_cV_z00_cutcomp.pdf}
	\caption{The marked correlation function for the concentration defined according to the velocity ratio. From top to bottom we show the results for the lowest mass cut in \simA, \simB, and \simC \ that excludes ill resolved halos. The shaded bands represent 2-sigma confidence regions generated by randomization of the marks. The dashed lines along the bottom denote the largest halo radius for a given value of the overdensity parameter. 
	\arz{Bottom panel here needs to be fixed so lines don't run into labels and virial radii markers and so on. 
	Label each panel by simulation and/or mass cut for $M_{200}$.}
	}
	\label{fig:cc_mcf_cV}
\end{figure}
%---------------------------------------------------

Moving on from concentrations, Figure~\ref{fig:cc_mcf_s} illustrates MCFs in which the mark is the shape parameter, $s$, of 
the halo. The environmental dependence of shape parameter is distinct from that of concentration in a number of ways. 
Notice that at all masses and at all halo definitions, less spherical halos (halos with smaller $s$) are more strongly 
clustered. Furthermore, decreasing $\Delta$ (increasing halo radius) only serves to increase this environmental 
dependence in all cases. This is indicative of halo shapes being driven by structure on scales larger than halo 
radii, in particular $R_{200}$, and is not entirely surprising given our significant amount of knowledge of the 
filamentary nature of large-scale structure. \arz{If possible, a line here with $\Delta \gg 200$. 
Seems like it could be instructive, let me know if this is going to take a long time or be very onerous.}


The spin MCFs are shown in Figure~\ref{fig:cc_mcf_spin}. Qualitatively, spin-dependent halo clustering is 
quite similar to shape-dependent halo clustering. Halos of low-spin cluster more strongly than halos of 
high-spin. \arz{Search the literature, particularly Brandon Allgood's papers and Andreas Faltenbacher whose 
names come to mind, for spin-dependent assembly bias. I'm trying to figure out if yours is the first paper 
to point this out.} Likewise, increasing halo radii by decreasing $\Delta$ in halo definitions only drives 
spin-dependent halo clustering to be stronger.

%---------------------------------------------------
\begin{figure}
	\centering
	\includegraphics[width=.4\textwidth]{all_mcf_s_z00_cutcomp.pdf}
	\caption{
	The marked correlation function for the shape of the halo. From top to bottom we show the results for the lowest mass cut in \simA, \simB, and \simC \ that excludes ill resolved halos. The shaded bands represent 2-sigma confidence regions generated by randomization of the marks. The dashed lines along the bottom denote the largest halo radius for a given value of the overdensity parameter.
	\arz{Bottom panel here needs to be fixed so lines don't run into labels and virial radii markers and so on. 
	Label each panel by simulation and/or mass cut for $M_{200}$.}
}
	\label{fig:cc_mcf_s}
\end{figure}
%--------------------------------------------


%---------------- spin MCF
\begin{figure}
	\centering
	\includegraphics[width=.4\textwidth]{all_mcf_spin_z00_cutcomp.pdf}
	\caption{The marked correlation function for the spin parameter of the halo. From top to bottom we show the results for the lowest mass cut in \simA, \simB, and \simC \ that excludes ill resolved halos. The shaded bands represent 2-sigma confidence regions generated by randomization of the marks. The dashed lines along the bottom denote the largest halo radius for a given value of the overdensity parameter.
	\arz{Bottom panel here needs to be fixed so lines don't run into labels and virial radii markers and so on. 
	Label each panel by simulation and/or mass cut for $M_{200}$.}
	}
	\label{fig:cc_mcf_spin}
\end{figure}
%-------------------------------


%----------------- satellite number MCF
\begin{figure}
	\centering
	\includegraphics[width=.4\textwidth]{all_mcf_nsat_z00_cutcomp.pdf}
	\caption{The marked correlation function for the number of satellites for a host halo. From top to bottom we show the results for the lowest mass cut in \simA, \simB, and \simC \ that excludes ill resolved halos. The shaded bands represent 2-sigma confidence regions generated by randomization of the marks. The dashed lines along the bottom denote the largest halo radius for a given value of the overdensity parameter.
	\arz{Many obvious cosmetic problems with this figure that need to be fixed.}
	}
	\label{fig:cc_mcf_nsat}
\end{figure}
%-------------------------------------------------------------------------------------------


The clustering of halos as a function of the number of satellite galaxies at fixed mass is of 
particular practical interest. Efforts to model survey data on the large-scale galaxy distribution, 
such as the halo occupation distribution (HOD) or conditional luminosity function (CLF) formalisms 
typically make the assumption that the multiplicity of satellite galaxies within a host dark matter 
halo depends solely upon halo mass. If this assumption is violated, then the phenomenological 
modeling of galaxy clustering can be more complicated than in these simplest scenarios.

Figure~\ref{fig:cc_mcf_nsat} shows clustering marked by satellite number as described in 
\S~\ref{section:methodology}. For all values of $\Delta$, halo clustering is strongly dependent 
upon satellite occupation at all masses. Interestingly, altering halo definitions as we have 
makes little difference to this dependence. In fact, defining halos to have larger radii (smaller 
$\Delta$) generally makes the environmental dependence of halo clustering more 
significant. Of course, our results pertain to satellite halos, or subhalos, rather than 
satellite galaxies, so the connection to observations and how one might 
model observed galaxy clustering is indirect, yet suggestive. 


%---------------------------------------------------------------------------------------------
\section{Discussion}
\label{section:discussion}
%---------------------------------------------------------------------------------------------

\arz{Repeating this comment here because of its importance. Look in the literature 
to see if spin-dependent halo clustering has been measured in the literature before. 
Look at http://arxiv.org/abs/1207.4476.}

We have shown (confirmed or reiterated may be a better word here) that for conventional halo definitions 
halo clustering strength is a strong function of ``auxiliary properties" halo concentration (either measured through a fit to the 
NFW profile or assigned non-parametrically as the ratio of the maximum circular velocity to the virial 
velocity), halo shape, halo spin, and number of subhalos 
for host halos over a wide range of masses. These findings are consistent 
with the now significant literature on the subject subject of halo assembly bias 
\arz{Reiterate citations here.}

We have explored the efficacy of alternative halo definitions to mitigate the dependence of halo 
clustering on these ``auxiliary properties." Rather generally, we find that these alternative definitions 
do {\em not} mitigate the effects of assembly bias. In most cases, defining halos to have significantly 
larger radii (lower $\Delta$) than in conventional halo definitions had only a modest influence on 
these assembly bias effects. Moreover, to the degree that these modified halo definitions had 
any effect at all, it was generally of the sense of 
making the assembly bias effect stronger, rather than weaker. 

One exception to this general conclusion is the case of halo concentration. 
Our results suggest that halo redefinition may be able to mitigate concentration 
dependent halo clustering. This is evident in Fig.~\ref{fig:cc_mcf_cnfw} and 
Fig.~\ref{fig:cc_mcf_cV}. Halo concentration is strongly correlated with halo formation 
time, so this suggests that such a redefinition may also aid in reducing assembly bias 
associated with halo formation time; however, this is a non-trivial extrapolation of our 
results and a follow-up study to assess halo formation times in alternative halo definitions 
is both interesting and warranted. 

Clearly, the halo definition that best mitigates 
concentration-dependent assembly bias must be mass dependent. Low 
values of $\Delta$ ($\Delta \sim 25$ with $R_{25} \sim 2$R_{200}$) seem appropriate near for our lowest 
mass-threshold sample (with $M_{200} \ge XXX$) whereas $\Delta \sim 200$ or slightly higher is adequate for 
our highest mass threshold sample (with $M_{200} \ge YYYY$). This result is reminiscent of much 
recent work on the so-called halo ``splashback radius." 
\arz{Here, compare our results to the splashback radius results.}


\arz{Here is where we could use ANOTHER FIGURE! What I would like to see is a visual 
comparison of our ``best" $\Delta$ as a function of halo mass threshold compared to 
the equivalent for the splashback radius. To be clear, it is trivial to go back and forth between 
$\Delta$ and radius (since radius $\propto \Delta^{-1/3}$) so this can be represented with 
either $\Delta$ or radius. If you choose to use radius, then it should probably be normalized, 
such as $R_{\Delta}/R_{200}$ and $R_{\rm splashback}/R_{200}$ and so on. This could 
be an important figure and point to future work on this subject.}


Overall, while some statistics of the halo can be made to become independent of environmental effects, there exist several that cannot. These environmental effects are not negligible when considering applications of the halo model. In contrast, if one is only interested in the concentration of the halo, it is feasible to take advantage of our methodology in order to remove environmental effects. In contrast, it is not possible to apply the halo model directly to statistics such as the number of satellite halos or the shape without accounting for these environmental effects. We do not yet have a full understanding of what leads to the visible effects that we see on some of these marks, but we present a series of cases in which an intuitive understanding may be gained as to the root causes.

We would also like to study our sample for some insight as to how this process of correction is actually functioning. For example, we would like to know if it is a more careful selection of halos in our simulation that results in the removal of this apparent assembly bias or if it is the increased noise in the statistics as a result of reducing the number of halos. There is also the possibility of halo statistics changing as a result of the new halo definitions to be considered. One can imagine that the addition of a large amount of mass toward the edge of the halo would reasonably change measures of concentration, shape, and spin. To test some of these possibilities, we run an algorithm to match halos across our multiple catalogs. We identify halos that are within a distance tolerance that we scale such that there are no identifications of multiple halos to the same object in another catalog. Catalogs are all matched against a single target catalog. This target catalog is chosen to fit the value of $\Delta$ that we find best removes assembly bias in large scales for the concentration marks chosen. For the \citet{diemer15} simulations, we find this to be a value of approximately $\Delta = 70$. Once the catalogs have been matched against this primary catalog, we rerun our marked correlation function calculations using only host halos identified across both catalogs. This tests if the exclusion of halos that are subsumed in the lower $\Delta$ catalog is the primary cause of our results as well as those halos that are commonly referred to as backsplash halos.

We can see the results of this test in Figure~\ref{fig:hvm_cfcompare} through Figure~\ref{fig:hvm_mcf_nsat}. Comparing in each case, the $\Delta = 200$ result when utilizing a catalog matched to our best fit $\Delta$ demonstrates less assembly bias than the full catalog for the mass cut would demonstrate for our concentration marks. While the magnitude of this result does vary with scale, upwards of 80\% of assembly bias can be accounted for as a result of this selection effect. Furthermore, most of the removed assembly bias occurs on scales that are comparable to the increase in the halo radius between the samples. This seems to indicate that there is significant contamination of host halo statistics by backsplash halos and other halos that would be subsumed into a common host given an increase in halo size.

\begin{figure*}
	\centering
	\includegraphics[width=.9\textwidth]{all_cfhilow_z00_hostsvmatch.pdf}
	\caption{The difference of the correlation function for only the top 20\% most concentrated halos and the bottom 20\% in concentration, normalized by the overall correlation function of the entire sample. The left (right) panel shows the ``mid mass'' (``matched'') cut on \simA \ and \simB. The ``matched'' cut accounts for potential backsplash halos, as discussed further in the text. The dashed lines along the bottom denote the largest halo radius for a given value of the overdensity parameter.}
	\label{fig:hvm_cfcompare}
\end{figure*}

\begin{figure*}
	\centering
	\includegraphics[width=.9\textwidth]{all_mcf_cnfw_z00_hostsvmatch.pdf}
	\caption{The marked correlation function for the concentration defined according to the NFW profile. The left (right) panel shows the ``mid mass'' (``matched'') cut on \simA \ and \simB. The ``matched'' cut accounts for potential backsplash halos, as discussed further in the text. The dashed lines along the bottom denote the largest halo radius for a given value of the overdensity parameter.}
	\label{fig:hvm_mcf_cnfw}
\end{figure*}

\begin{figure*}
	\centering
	\includegraphics[width=.9\textwidth]{all_mcf_cV_z00_hostsvmatch.pdf}
	\caption{The marked correlation function for the concentration defined according to the velocity ratio. The left (right) panel shows the ``mid mass'' (``matched'') cut on \simA \ and \simB. The ``matched'' cut accounts for potential backsplash halos, as discussed further in the text. The dashed lines along the bottom denote the largest halo radius for a given value of the overdensity parameter.}
	\label{fig:hvm_mcf_cV}
\end{figure*}

\begin{figure*}
	\centering
	\includegraphics[width=.9\textwidth]{all_mcf_s_z00_hostsvmatch.pdf}
	\caption{The marked correlation function for the shape parameter. The left (right) panel shows the ``mid mass'' (``matched'') cut on \simA \ and \simB. The ``matched'' cut accounts for potential backsplash halos, as discussed further in the text. The dashed lines along the bottom denote the largest halo radius for a given value of the overdensity parameter.}
	\label{fig:hvm_mcf_s}
\end{figure*}

\begin{figure*}
	\centering
	\includegraphics[width=.9\textwidth]{all_mcf_spin_z00_hostsvmatch.pdf}
	\caption{The marked correlation function for the spin parameter. The left (right) panel shows the ``mid mass'' (``matched'') cut on \simA \ and \simB. The ``matched'' cut accounts for potential backsplash halos, as discussed further in the text. The dashed lines along the bottom denote the largest halo radius for a given value of the overdensity parameter.}
	\label{fig:hvm_mcf_spin}
\end{figure*}

\begin{figure*}
	\centering
	\includegraphics[width=.9\textwidth]{all_mcf_nsat_z00_hostsvmatch.pdf}
	\caption{The marked correlation function for the satellite number. The left (right) panel shows the ``mid mass'' (``matched'') cut on \simA \ and \simB. The ``matched'' cut accounts for potential backsplash halos, as discussed further in the text. The dashed lines along the bottom denote the largest halo radius for a given value of the overdensity parameter.}
	\label{fig:hvm_mcf_nsat}
\end{figure*}

However, it is worth noting that statistics such as spin parameter, shape, and satellite numbers do not benefit from this selection effect. Assembly bias in these statistics remains nearly identical, suggesting that the selection effects that preferentially account for backsplash halos do not impact these marks. This does serve as a consistency check: given that our methodology does not seem to have a positive impact on the behavior of these marks, the fact that this selection criteria would have positive impact on the data would be an oddity. Instead, we have another confirmation as to the intrinsic ties between these halo statistics and environment.

Given that the removal of assembly bias is not being driven by introducing noice, we now will attempt to determine why the different marks exhibit different behaviors. The first interesting feature is how well our two separate definitions of concentration interact with each other. In the case that the halo can be described well by an NFW profile, one expects a direct relationship between the NFW defined halo concentration and the velocity ratio defined halo concentration. While some variation can be expected due to halos not perfectly being fit by an NFW profile, we do see that the features in one concentration proxy are mirrored in the other. This allows for the two concentration markers to support each other well with regards to our ability to remove environmental effects on large scales.

Halo shape and satellite number are statistics that do not end up having their environmental effects removed and can even be made more prominent by our methodology. One intuitive way to consider the former statistic is in the context of the cosmic web. Studies have shown a statistically significant alignment between filaments and satellite galaxy position \citep{tempel15, velliscig15}. Our method then expands the halo radius and subsumes material that was previously outside of the halo. A simple graphic illustrates this potential effect in Figure~\ref{fig:plotcircles}. As there is a preferential distribution of these satellites that are being subsumed into the halo, this would serve to induce a shape to the halo that would then be determined by Rockstar. In addition, as our satellite number is chosen by those halos within the halo radius, we anticipate that the most clustered regions would see the largest increase in satellite count and thus see an increase in the satellite number mark.

The ``sweet spot'' behavior of this method is also of interest to us. The halo redefinition process that we use serves to decrease halo clustering for the most concentrated halos and increase halo clustering for the least concentrated halos. In the case of the high concentration cut, the reduced clustering can be seen as a result of halo exclusion. As we exchange collections of tightly clustered smaller halos for a single larger halo, then the most prominent contribution of the two-point correlation function will be reduced at all scales. Furthermore, it suggests that the assembly bias that we witness is correlated with the choice of halo definition. This effect may be necessary to explaining the mixed results in the field at determining just how strong assembly bias is.

\begin{figure*}
	\centering
	\includegraphics[width=.9\textwidth]{plotcircles_coolwarm.pdf}
	\caption{A $20 \hMpc$ deep cut of \simB \ along the z-axis. This zoom-in demonstrates the process that decreases the shape parameter as a function of clustering. The size of each circle represents the projection of a spherical dark matter halo with a given halo radius onto the x-y plane. Filled circles use the $\Delta = 200$ catalog and unfilled circles use the $\Delta = 10$ catalog in order to make the effect more visible. Color scale refers to the log shape mark, normalized by halos of the same mass.}
	\label{fig:plotcircles}
\end{figure*}

%% another thing to repeat on one of the Diemer boxes. High priority.

Given the differences between the simulation results and with the data available to us with the \citet{diemer15} simulations, we then explore the effect of the choice in mass cut. The low mass cut effects are shown in the Figure~\ref{fig:hvl_cfcompare} through Figure~\ref{fig:hvl_mcf_nsat}. This cut explores the area in which we are including ill-resolved halos potentially into the simulation. Namely, utilizing this mass cut on \simB \ data includes halos that do not fit the form of the expected monotonic halo mass-concentration relation. We can determine several facts from this exercise in resolution testing. The first is that the general behavior of the marks seems to be the same regardless of the influence of the resolution effects. Decreasing the value of $\Delta$ still moves our marked correlation functions in the same directions as the previous mass cut, although often from a very different starting location. Even with potentially ill resolved objects, the method can be brought to bear upon the problem - potentially a concern if the simulation's ability to resolve a particular statistic is questionable. Also noteable is that the level of the assembly bias is reduced in the case of this lower mass cut in \simB \ when compared to \simA. This is distinctly different from our previous example, in which both \simA \ and \simB \ both contain only well resolved halos within the mass cut and have nearly identical assembly bias at every scale, aside from simulation noise. This seems to imply that including unphysical halos in our sample makes it difficult to determine the actual assembly bias at work.

\begin{figure*}
	\centering
	\includegraphics[width=.9\textwidth]{all_cfhilow_z00_hostsvlow.pdf}
	\caption{The difference of the correlation function for only the top 20\% most concentrated halos and the bottom 20\% in concentration, normalized by the overall correlation function of the entire sample. The top row uses \simA \  data while the bottom row uses \simB \ data. The left column utilizes the ``mid mass'' cutoff, while the right column demonstrates the ``low mass'' cutoff. The dashed lines along the bottom denote the largest halo radius for a given value of the overdensity parameter}
	\label{fig:hvl_cfcompare}
\end{figure*}

\begin{figure*}
	\centering
	\includegraphics[width=.9\textwidth]{all_mcf_cnfw_z00_hostsvlow.pdf}
	\caption{Comparison of the marked correlation function for the concentration defined according to the NFW profile between the ``mid mass'' cutoff (left column) and the ``low mass'' cutoff (right column). The top row uses \simA \ data while the bottom row uses \simB \ data. The shaded bands represent 2-sigma confidence regions generated by randomization of the marks. The dashed lines along the bottom denote the largest halo radius for a given value of the overdensity parameter.}
	\label{fig:hvl_mcf_cnfw}
\end{figure*}

\begin{figure*}
	\centering
	\includegraphics[width=.9\textwidth]{all_mcf_cV_z00_hostsvlow.pdf}
	\caption{Comparison of the marked correlation function for the concentration defined according to the velocity ratio between the ``mid mass'' cutoff (left column) and the ``low mass'' cutoff (right column). The top row uses \simA \ data while the bottom row uses \simB \ data. The shaded bands represent 2-sigma confidence regions generated by randomization of the marks. The dashed lines along the bottom denote the largest halo radius for a given value of the overdensity parameter.}
	\label{fig:hvl_mcf_cV}
\end{figure*}

\begin{figure*}
	\centering
	\includegraphics[width=.9\textwidth]{all_mcf_s_z00_hostsvlow.pdf}
	\caption{Comparison of the marked correlation function for the shape of the halo between the ``mid mass'' cutoff (left column) and the ``low mass'' cutoff (right column). The top row uses \simA \ data while the bottom row uses \simB \ data. The shaded bands represent 2-sigma confidence regions generated by randomization of the marks. The dashed lines along the bottom denote the largest halo radius for a given value of the overdensity parameter.}
	\label{fig:hvl_mcf_s}
\end{figure*}

\begin{figure*}
	\centering
	\includegraphics[width=.9\textwidth]{all_mcf_spin_z00_hostsvlow.pdf}
	\caption{Comparison of the marked correlation function for the spin of the halo between the ``mid mass'' cutoff (left column) and the ``low mass'' cutoff (right column). The top row uses \simA \ data while the bottom row uses \simB \ data. The shaded bands represent 2-sigma confidence regions generated by randomization of the marks. The dashed lines along the bottom denote the largest halo radius for a given value of the overdensity parameter.}
	\label{fig:hvl_mcf_spin}
\end{figure*}

\begin{figure*}
	\centering
	\includegraphics[width=.9\textwidth]{all_mcf_nsat_z00_hostsvlow.pdf}
	\caption{Comparison of the marked correlation function for the satellite number between the ``mid mass'' cutoff (left column) and the ``low mass'' cutoff (right column). The top row uses \simA data while the bottom row uses \simB data. The shaded bands represent 2-sigma confidence regions generated by randomization of the marks. The dashed lines along the bottom denote the largest halo radius for a given value of the overdensity parameter.}
	\label{fig:hvl_mcf_nsat}
\end{figure*}

We can repeat this exercise by looking at a different set of mass cuts. In this case we will utilize our highest mass cut on \simB and \simC. This case does not include poorly resolved halos as we were including in the previous example. Instead, the smaller sized box runs the risk of having less of the most massive halos, resulting in having a larger variance for objects at high mass. The results of this are shown in Figure~\ref{fig:hvh_cfcompare} through Figure~\ref{fig:hvh_mcf_nsat}. There are several key observations to be taken away from this series of plots. The first is the fact that like in our previous example, we see no significant change in assembly bias when only including well resolved halos and see a considerable change when we include unresolved halos in \simC. Of greater import is the fact that we see that in several cases, when interest is only in the most massive halos we find that there is next to no assembly bias. Changing our definition of the halo radius in these cases can even induce halo assembly bias. Given the previous discussion regarding the differences in halo radius definitions chosen (e.g. critical density versus mean density), this implies that we can have very different measurements of this effect solely based on halo definition - a fact that has yet to be explored thoroughly within the literature.

\begin{figure*}
	\centering
	\includegraphics[width=.9\textwidth]{all_cfhilow_z00_hostsvhigh.pdf}
	\caption{The difference of the correlation function for only the top 20\% most concentrated halos and the bottom 20\% in concentration, normalized by the overall correlation function of the entire sample. The top row uses \simB \ data while the bottom row uses \simC \ data. The left column utilizes the ``mid mass'' cutoff, while the right column demonstrates the ``high mass'' cutoff. The dashed lines along the bottom denote the largest halo radius for a given value of the overdensity parameter}
	\label{fig:hvh_cfcompare}
\end{figure*}

\begin{figure*}
	\centering
	\includegraphics[width=.9\textwidth]{all_mcf_cnfw_z00_hostsvhigh.pdf}
	\caption{Comparison of the marked correlation function for the concentration defined according to the NFW profile between the ``mid mass'' cutoff (left column) and the ``high mass'' cutoff (right column). The top row uses \simB \ data while the bottom row uses \simC \ data. The shaded bands represent 2-sigma confidence regions generated by randomization of the marks. The dashed lines along the bottom denote the largest halo radius for a given value of the overdensity parameter.}
	\label{fig:hvh_mcf_cnfw}
\end{figure*}

\begin{figure*}
	\centering
	\includegraphics[width=.9\textwidth]{all_mcf_cV_z00_hostsvhigh.pdf}
	\caption{Comparison of the marked correlation function for the concentration defined according to the velocity ratio between the ``mid mass'' cutoff (left column) and the ``high mass'' cutoff (right column). The top row uses \simB \ data while the bottom row uses \simC \ data. The shaded bands represent 2-sigma confidence regions generated by randomization of the marks. The dashed lines along the bottom denote the largest halo radius for a given value of the overdensity parameter.}
	\label{fig:hvh_mcf_cV}
\end{figure*}

\begin{figure*}
	\centering
	\includegraphics[width=.9\textwidth]{all_mcf_s_z00_hostsvhigh.pdf}
	\caption{Comparison of the marked correlation function for the shape of the halo between the ``mid mass'' cutoff (left column) and the ``high mass'' cutoff (right column). The top row uses \simB \ data while the bottom row uses \simC \ data. The shaded bands represent 2-sigma confidence regions generated by randomization of the marks. The dashed lines along the bottom denote the largest halo radius for a given value of the overdensity parameter.}
	\label{fig:hvh_mcf_s}
\end{figure*}

\begin{figure*}
	\centering
	\includegraphics[width=.9\textwidth]{all_mcf_spin_z00_hostsvhigh.pdf}
	\caption{Comparison of the marked correlation function for the spin of the halo between the ``mid mass'' cutoff (left column) and the ``high mass'' cutoff (right column). The top row uses \simB \ data while the bottom row uses \simC \ data. The shaded bands represent 2-sigma confidence regions generated by randomization of the marks. The dashed lines along the bottom denote the largest halo radius for a given value of the overdensity parameter.}
	\label{fig:hvh_mcf_spin}
\end{figure*}

\begin{figure*}
	\centering
	\includegraphics[width=.9\textwidth]{all_mcf_nsat_z00_hostsvhigh.pdf}
	\caption{Comparison of the marked correlation function for the satellite number between the ``mid mass'' cutoff (left column) and the ``high mass'' cutoff (right column). The top row uses \simB \ data while the bottom row uses \simC \ data. The shaded bands represent 2-sigma confidence regions generated by randomization of the marks. The dashed lines along the bottom denote the largest halo radius for a given value of the overdensity parameter.}
	\label{fig:hvh_mcf_nsat}
\end{figure*}

%----------------------
\section[]{Conclusions}
\label{section:conclusions}
%----------------------

We have looked at how to use CFs and MCFs in order to analyze the environmental effects upon the properties of the halo. We have suggested a method of removing the mass dependence that is not subject to the small number statistics at large halo masses. Taking our various tests, we then apply a change to the threshold density $\Delta$ in an attempt to remove the effect that environment has upon these properties. We come to the following conclusions from our simulation data.

\begin{itemize}
	\item Our halo redefinition method does not cause any substantial breakdown in the ROCKSTAR halo finding algorithm, though this may not be the case for every halo finding methodology. This is something that should be considered prior to utilization of this method, unless working directly from particle data. As our initial halo sizes and locations are determined through spherical overdensities, it cannot be assumed that starting from a FoF grouping and then determining values through particle data directly will produce identical results. Similarly, different cosmologies may remove environmental effects at different scales.

	\item When looking at the two-point correlation function, there appears to be a ``sweet spot'' that appears to remove environmental effects the most efficiently. Going beyond that seems to reintroduce environmental effects, possibly as an extreme side effect of halo exclusion.

	\item For our marked correlation functions we see that both proxies of concentration that we use as marks show significant removal of environmental effects at large scales for similar values of the overdensity parameter $\Delta$. In cases where one is only interested in the concentration of dark matter halos and large scales (or correspondingly small values of k), this method will allow you to compensate for bias that environment could introduce to calculations dependent upon the halo model. This may prove valuable for calculations such as that of the shear power spectrum calculated through weak lensing.

	\item The environmental effects on the shape of the host halo and the satellite number of the host halo cannot be removed regardless of the chosen redefinition of $\Delta$. We propose that this may be intrinsically tied to the nature of the filaments, whose effects cannot be removed by a simple redefinition of the halo radius.

	\item This method is definitively related to the mass of the halos that are being observed. Furthermore, it appears that the majority of the reduction in assembly bias is tied to the exclusion of halos from the catalog as a result of being subsumed into larger halos. This information does not seem to be contradictory; it can be intuitively understood that the region about the most massive halos will be different than the region around the least massive halos, leading to a different frequency at which halos are being excluded. It does however warrant that careful consideration be given to the sample of halos that are of interest.

	\item The selection of halo size is intrinsically related to the assembly bias and varies across scales. This might help to resolve contradictory results in the search for halo assembly bias in the literature.
\end{itemize}

This methodology, while certainly not perfect in accounting for assembly bias, may be of significance when applied to galaxy formation models and give insight into seemingly conflicting results. Provided that the properties of interest in a given model behave well under our redefinition, it will allow us to create better mock galaxy catalogs without resorting to more complicated models requiring halo formation histories - giving us another powerful tool to test observation against.

There remain possible uncertainties to study in the future. One possible area of follow-up is the matter of simulation cosmology, which is not explored in this text. It is possible that the choice of cosmology may change observed assembly bias as a function of the halo masses, something that our methodology should be capable of observing. Furthermore, we can determine if the choice of halo size that best reduces assembly bias is a function of the chosen cosmology. This may be of interest in attempting to determine signatures of assembly bias in observational samples in the future.

\section*{Acknowledgments}

We are grateful to many people.

%%\bibliographystyle{plainnat}
\bibliography{master}

\section*{Appendix}
\label{section:appendix_massres}

One natural question that might arise in the analysis of this work is the nature of the resulting assembly bias trends. Our focus in the main sections of this paper is on the nature of the assembly bias changing as a function of the mass cut chosen. Our conclusions include the fact that there is a strong mass dependence on halo assembly bias that must be accounted for seperately depending on the halos of interest in a study. However, while the existence of this trend is clear within our analysis, the determination that this is solely due to the masses of the halos included in our calculation is less clear upon closer inspection. One possibility that might be particularly concerning is the potential that the different simulations have created halos that have fundamentally different clustering and this is leading to the result that we are interpreting as a mass dependence on assembly bias. Thankfully, though our statistics become less meaningful to carry out this calculation, we can carry out a comparison using the same mass cut across two of our simulations, knowing that these will only contain well resolved halos.

While not addressed directly, Figure~\ref{fig:hvl_cfcompare} through Figure~\ref{fig:hvl_mcf_nsat} contain a demonstration of the result that we are seeking in the left column of panels. The lower left panels show various marks of interest for \simB \ using the ``mid mass'' cut on the data set. In comparison, the upper left panel contains the same marks of interest for the \simA \ using the same mass cut. In the latter, there are fewer halos in this mass cut range, as a result of the smaller simulation box size. However, we note that despite the additional noise in the data set, the behavior of the assembly bias measurement is nearly identical within tolerances accounting for differences between simulations and noise. This motivates our conclusion that the driver behind the behavior is the mass cut of the data sets rather than the resolution of the simulation.

\label{lastpage}

\end{document}
